\documentclass{fkssolpub}

\usepackage[czech]{babel}
\usepackage{fontspec}
\usepackage{fkssugar}
\usepackage{amsmath}
\usepackage{graphicx}

\author{Ondřej Sedláček}
\school{Gymnázium Oty Pavla} 
\series{74-I}
\problem{1} 

\begin{document}

Víme ze zadání, že pro čísla $a$, $b$ platí rovnice:

\[
	a^2 + b = b^2 + a
\]

Z této rovnice můžeme vyjádřit hodnotu $b$ na základě $a$:

\[
	a^2 - a = b^2 - b
\]
\[
	a (a - 1) = b (b - 1)
\]

Z tohoto tvaru je již zřejmě vidět, že řešením jsou jedno z $b \in \{a, 1 - a\}.$ Teď musíme tedy zjistit, jaký z výrazů $a^2 + a$, $a^2 - a + 1$ nabývá menších hodnot. Stačí nám tedy převést tyto výrazy do vrcholového tvaru:

\[
	a^2 + a = a^2 + a + \frac{1}{4} - \frac{1}{4} = \left(a + \frac{1}{2}\right)^2 - \frac{1}{4}
\]
\[
	a^2 - a + 1 = a^2 - a + \frac{1}{4} + \frac{3}{4} = \left(a - \frac{1}{2}\right)^2 + \frac{3}{4}
\]

Odtud je zřejmě vidět, že nejmenší hodnoty nabývají výrazy ze zadání pro $a = b = -\frac{1}{2}
$, kdy $a^2 + b = b^2 + a = -\frac{1}{4}$, čímž jsme nalezli řešení.
\end{document}
