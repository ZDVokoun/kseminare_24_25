\documentclass{fkssolpub}

\usepackage[czech]{babel}
\usepackage{fontspec}
\usepackage{fkssugar}
\usepackage{amsmath}
\usepackage{graphicx}

\newcommand{\dd}{\mathrm{d}}
\renewcommand{\angle}{\sphericalangle}

\author{Ondřej Sedláček}
\school{Gymnázium Oty Pavla} 
\series{6}
\problem{A} 

\begin{document}

Jako první dokážu, že když platí $abc = 1$, tak je trojice záhadná. Tehdy cyklicky platí, že $a^2 = \frac{1}{b^2}{c^2}$, což když dosadíme, dostaneme:

\[
  \sqrt{a^2 + \frac{1}{a^2 c^2} + 2ab} + \sqrt{b^2 + \frac{1}{b^2 a^2} + 2bc} + \sqrt{c^2 + \frac{1}{c^2 b^2} + 2ca} =
\]
\[
  = \sqrt{a^2 + b^2 + 2ab} + \sqrt{b^2 + c^2 + 2bc} + \sqrt{c^2 + a^2 + 2ca} = a + b + b + c + c + a = 2(a + b + c)
\]

Teď sporem dokážu, trojice je záhadná právě tehdy, když $abc = 1$. Předpokládejme tedy BÚNO, že $abc < 1$. Protože čísla $a,b,c$ jsou kladná čísla a kvadratická funkce a odmocnina jsou v kladných číslech definovány a rostoucí, platí:

\[
  a < \frac{1}{bc}
\]
\[
  c^2 + a^2 + 2ca < c^2 + \frac{1}{c^2 b^2} + 2ca
\]
\[
  \sqrt{c^2 + a^2 + 2ca} < \sqrt{c^2 + \frac{1}{c^2 b^2} + 2ca}
\]

Analogicky pak můžeme získat zbývající dvě nerovnosti, které když sečteme, dostaneme:

\[
  \sqrt{a^2 + \frac{1}{a^2 c^2} + 2ab} + \sqrt{b^2 + \frac{1}{b^2 a^2} + 2bc} + \sqrt{c^2 + \frac{1}{c^2 b^2} + 2ca} >
\]

\[
  > \sqrt{a^2 + b^2 + 2ab} + \sqrt{b^2 + c^2 + 2bc} + \sqrt{c^2 + a^2 + 2ca} = 2(a + b + c)
\]

Vidíme tedy, že rovnost ze zadání v tomto případě nikdy nemůže platit.

Protože v případě $abc > 1$ je postup podobný, dokázali jsme, že trojice je záhadná právě tehdy, když $abc = 1$. A protože $abc = cba = 1$, trojice $(c,b,a)$ je taky záhadná, což jsme chtěli dokázat.


\end{document}
