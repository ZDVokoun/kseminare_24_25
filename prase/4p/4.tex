\documentclass{fkssolpub}

\usepackage[czech]{babel}
\usepackage{fontspec}
\usepackage{fkssugar}
\usepackage{amsmath}
\usepackage{graphicx}

\author{Ondřej Sedláček}
\school{Gymnázium Oty Pavla} 
\series{4p}
\problem{4} 

\begin{document}

Firstly, we are going to permute the digits of the number $n$ in such a way so the digits 0,1,2,3 are at the least significant positions. If we then prove that by permuting these digits at the least significant places we can generate all remainders after division by 7, we will prove the claim in the assignment.

It holds that:

\[
  10^3 \cdot a_3 + 10^2 \cdot a_2 + 10 \cdot a_1 + a_0 \equiv 6a_3 + 2a_2 + 3a_1 + a_0 \pmod{7}
\]

Where numbers $a_0, a_1, a_2, a_3$ is permutation of $\{0,1,2,3\}$. Therefore, we can write out the permutations generating all the remainders:

\[
 6 \cdot 0 + 2 \cdot 2 + 3 \cdot 3 + 1 \equiv 0 \pmod{7}
\]
\[
 6 \cdot 1 + 2 \cdot 0 + 3 \cdot 2 + 3 \equiv 1 \pmod{7}
\]
\[
 6 \cdot 2 + 2 \cdot 1 + 3 \cdot 3 + 0 \equiv 2 \pmod{7}
\]
\[
 6 \cdot 0 + 2 \cdot 2 + 3 \cdot 1 + 3 \equiv 3 \pmod{7}
\]
\[
 6 \cdot 0 + 2 \cdot 1 + 3 \cdot 2 + 3 \equiv 4 \pmod{7}
\]
\[
 6 \cdot 1 + 2 \cdot 2 + 3 \cdot 3 + 0 \equiv 5 \pmod{7}
\]
\[
 6 \cdot 0 + 2 \cdot 1 + 3 \cdot 3 + 2 \equiv 6 \pmod{7}
\]

As we can see, it is possible to generate all remainders by permuting these digits. Thus, we are always able to permute these digits so as to make the number $n$ divisible by 7.

\end{document}
