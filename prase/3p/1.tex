\documentclass{fkssolpub}

\usepackage[czech]{babel}
\usepackage{fontspec}
\usepackage{fkssugar}
\usepackage{amsmath}
\usepackage{graphicx}

\author{Ondřej Sedláček}
\school{Gymnázium Oty Pavla} 
\series{3p}
\problem{1} 

\begin{document}

Jako první vyjádříme funkční hodnotu $f(n)$ modulo pěti a modulo devíti:

\[
	f(n) \equiv n \pmod{5}
\]
\[
	f(n) \equiv 1 + 2 + \dots + n \equiv \frac{n (n + 1)}{2} \pmod{9}
\]

Z těchto kongruencí víme, že aby $f(n)$ bylo násobkem 45, musí $5 \vert n$ a buď $9 \vert n$ nebo $n \equiv 8 \pmod{9}$. Nejmenší přirozené číslo, které tyto podmínky splňuje, je $n = 35$. Tím jsme našli řešení úlohy.

\end{document}
