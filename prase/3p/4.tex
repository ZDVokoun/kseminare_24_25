\documentclass{fkssolpub}

\usepackage[czech]{babel}
\usepackage{fontspec}
\usepackage{fkssugar}
\usepackage{amsmath}
\usepackage{graphicx}

\author{Ondřej Sedláček}
\school{Gymnázium Oty Pavla} 
\series{3p}
\problem{4} 

\begin{document}

Když do vzorce, který má pro funkci $f: \mathbb{P} \rightarrow \mathbb{P}$ platit, dosadíme dvě stejná prvočísla $p$, dostaneme:

\[
	\text{NSD}(p, p) = p = \text{NSD}(f^p(p), f^p(p)) = f^p(p)
\]

Tedy pro každé prvočíslo $p$ musí platit, že $f^p(p) = p$.

Teď předpokládejme, že $f(p) = x$. Protože nutně platí, že $f^x(x) = x$, platí pak:

\[
	f^{p - 1}(f(p)) = f^{p - 1}(x) = p
\]
\[
	x = f(p) = f(f^{p - 1}(x)) = f^p(x) = f^x(x)
\]

Pokud $p = x$, pak podmínka zjevně platí a získáme z toho předpis funkce $f(p) = p$. Teď dokážu, že jenom tato funkce umí splnit tyto podmínky.

Předpokládejme, že $p \neq x$. Pak víme, že platí $\text{NSD}(p, x) = 1$ a proto když do rovnice $f^p(x) = f^x(x)$ budeme postupně dosazovat z jedné strany do druhé na principu Euklidova algoritmu, dostaneme $f(x) = x$. To je však ve sporu s tím, že $f^p(p) = p$, protože by z toho vyplývalo $f^p(p) = f^{p - 1}(f(p)) = f^{p - 1}(x) = x$.

Tím jsme tedy dokázali, že jediná platná funkce je $f(p) = p$. Q. E. D.

\end{document}
