\documentclass{fkssolpub}

\usepackage[czech]{babel}
\usepackage{fontspec}
\usepackage{fkssugar}
\usepackage{amsmath}
\usepackage{graphicx}

\author{Ondřej Sedláček}
\school{Gymnázium Oty Pavla} 
\series{3p}
\problem{7} 

\begin{document}

Jako první ukážeme, že neexistuje $n$ a $k$, pro které $n = f^k(n)$. Tehdy pak nutně platí, že $n = f^k(n) = f^{2k}(n) = f^{3k}(n) = \dots$, a bylo by nekonečně mnoho $k \in \mathbb{N}$ splňující $f^k(n) \leq n + k + 1$. Toto ale porušuje podmínku ze zadání, tudíž takové $n$ a $k$ nemůže existovat.

Dále ukážeme, že neexistuje $n$, pro který $f(n) < n$. Tehdy je nutně $f(n) \leq n + 2$, tedy pro všechna $k > 1$ platí $f^k(n) > n + k + 1$. Máme tedy splněnou podmínky pro $n$, ale snadno ukážeme, že podmínky nelze splnit pro $f(n)$:

\[
	f^k(n) = f^{k - 1}(f(n)) > n + k + 1 > f(n) + (k - 1) + 1
\]

Z této nerovnosti je zřejmé, že pro $f(n)$ jsou všechna $f^k(f(n)) > f(n) + k + 1$, což je ale spor s podmínkou ze zadání, čímž jsme dokázali, že neexistuje $n$, pro které $f(n) < n$.

Z toho už víme, že posloupnost $n, f(n), f^2(n), \dots$ je rostoucí. Toho využijeme k důkazu, že pro každé $n$ je jediným řešením $f^k(n) \leq n + k + 1$ číslo $k = 1$. Pokud by to totiž neplatilo, pak zároveň $f^{k - 1}(n) > n + k$, což z toho spolu s tím, že $f^{k - 1}(n) < f^k(n)$, vyplývá:

\[
	n + k < f^{k - 1}(n) < f^k(n) \leq n + k + 1
\]

V přirozených číslech tato nerovnice nemá řešení, proto tedy jediným řešením $f^k(n) \leq n + k + 1$ je číslo $k = 1$.

Teď máme dost informací na to, abychom odvodili předpis funkcí $f$. Víme, že pro každé $n$ platí $f(n) \leq n + 2$, $f^2(n) > n + 3$, $f(f(n)) \leq f(n) + 2$. Z toho zjistíme funkci $f^2(n)$:

\[
	n + 3 < f^2(n) \leq f(n) + 2 \leq n + 4
\]

Tím pádem $f^2(n) = n + 4$. Teď už potřebujeme jen zjistit $f(n)$:

\[
	f^2(n) - 2 \leq f(n) \leq n + 2
\]

A proto jediná funkce, která splňuje zadání, je $f(n) = n + 2$. Tím je tedy důkaz u konce.

\end{document}
