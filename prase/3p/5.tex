\documentclass{fkssolpub}

\usepackage[czech]{babel}
\usepackage{fontspec}
\usepackage{fkssugar}
\usepackage{amsmath}
\usepackage{graphicx}

\author{Ondřej Sedláček}
\school{Gymnázium Oty Pavla} 
\series{3p}
\problem{5} 

\begin{document}

Protože funkce $f$ dělí sudá čísla, dokud se z nich nestanou čísla lichá, má smysl uvažovat jen lichá $k$. Dále budu místo čísla $n$ pracovat s lichým číslem $n'$, pro který platí vztah $n = 2^x \cdot n'$, kde $x \in \mathbb{Z}_0^{+}$.

Teď budu chtít ukázat, že existuje takové číslo $k$ a $y$, že:

\[
	3 n' k + 1 = 2^y
\]

Pokud takové číslo $k$ a $y$ existuje, pak v posloupnosti dostaneme mocninu dvojky, která se posléze vykrátí až na jedničku. Takovou rovnici nemůže sudé $k$ splnit, tím pádem jakmile začneme posloupnost číslem $kn$, tak jako první liché číslo v posloupnosti dostaneme $k n'$. Stačí nám tedy ukázat, že existuje $y$ takové, že $3 n' \vert 2^y - 1$.

Toto si nejprve převedeme na kongruenci a upravíme:

\[
	2^y - 1 \equiv 0 \pmod{3n'}
\]
\[
	2^y \equiv 1 \pmod{3n'}
\]

Protože víme, že 2 a $3n'$ jsou nesoudělná čísla, z Eulerovy věty pak platí, že tuto kongruenci splňuje $y = \phi(3n')$. Tím jsme dokázali, že vždy existuje číslo $k$ a $y$, které rovnici výše splňuje.

\end{document}
