\documentclass{fkssolpub}

\usepackage[czech]{babel}
\usepackage{fontspec}
\usepackage{fkssugar}
\usepackage{amsmath}
\usepackage{graphicx}

\author{Ondřej Sedláček}
\school{Gymnázium Oty Pavla} 
\series{3p}
\problem{6} 

\begin{document}

Pro nedegenerovaný trojúhelník platí, že:

\[
	\max(a,b,c) \leq s
\]

kde $s = \frac{a + b + c}{2}$. Důležité je si tedy všimnout, že $s$ se při vytváření nových trojúhelníků nemění:

\[
	s = \frac{(a + b - c) + (a + c - b) + (b + c - a)}{2} = \frac{a + b + c}{2}
\]

Číslo $s$ tedy zůstává konstantní, proto budeme chtít určit typ trojúhelníku, u kterého nejdelší strana nebude růst.

Bez újmy na obecnosti předpokládejme, že $a \geq b \geq c$. Pak nám vychází čtyři možné případy, které nám mohou pro trojúhelníky nastat, a to $a > b > c$, $a = b = c$, $a > b = c$, $a = b > c$.

Začneme nejprve případem, kdy $a > b > c$. Uspořádání nových hran pak bude:

\[
	a + b - c > a + c - b > b + c - a
\]

Z toho víme, že nová nejdelší strana má délku $a + b - c > a$, tedy nejdelší strana se nám prodlužuje. Protože trojúhelník zůstal různostranný, jeho nejdelší strana se bude pořád prodlužovat, proto jednou zdegeneruje.

Dalším případem je $a = b = c$. Pak zase dostaneme rovnostranný trojúhelník, protože:

\[
	a + b - c = a + c - b = b + c - a = a = b = c
\]

Dostaneme tedy tentýž trojúhelník, takže ten můžeme kreslit do nekonečna.

Teď nám zbývají poslední dva případy, kdy máme rovnoramenné trojúhelníky. Začneme nejprve případem, kdy $a = b > c$. Tehdy se uspořádání změní:

\[
	a + b - c > c = a + c - b = b + c - a
\]

V tomto případě se nejdelší strana prodlouží, ale změní se nám uspořádání. To vyřešíme tím, že rozebereme poslední případ, který má totožné uspořádání, a to $a > b = c$:

\[
	a = a + b - c = a + c - b > b + c - a
\]

Tehdy nejdelší hrana zůstane stejně dlouhá, ale změní se uspořádání na případ, kdy $a = b > c$, při kterým nejdelší hrana se prodlouží. Proto i pro rovnoramenné trojúhelníky nelze kreslit trojúhelníky do nekonečna.

Tím jsme tedy ukázali, že do nekonečna může kreslit jen rovnostranné trojúhelníky. Q. E. D.

\end{document}
