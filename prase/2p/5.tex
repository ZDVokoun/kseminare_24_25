\documentclass{fkssolpub}

\usepackage[czech]{babel}
\usepackage{fontspec}
\usepackage{fkssugar}
\usepackage{amsmath}
\usepackage{graphicx}

\author{Ondřej Sedláček}
\school{Gymnázium Oty Pavla} 
\series{2p}
\problem{5} 

\begin{document}

Ze soustavy trojúhelníkových nerovností platí, že pokud je trojúhelník o stranách $a$, $b$, $c$ nedegenerovaný, pak je obsah daného trojúhelníku kladné reálné číslo nebo jinak zapsáno:

\[
	(a + b + c)(- a + b + c)(a - b + c)(a + b - c) > 0
\]

Nechť je výraz na levé straně $g(a, b, c)$. Pak platí:

\[
	g(a, b, c) = (a + b + c)(- a + b + c)(a - b + c)(a + b - c) = -a^4 - b^4 -c^4 + 2a^2 b^2 + 2b^2 c^2 + 2a^2 c^2
\]

Teď dokážeme, že dokážeme hrany čtyřstěnu rozdělit na dvě trojice podle zadání. Nechť hrany $a$, $b$, $c$ tvoří podstavu čtyřstěnu a $d$,$e$,$f$ jsou postupně hrany protější vůči zbývajícím třem. Protože stěny čtyřstěnu jsou nutně nedegenerované trojúhelníky, platí:

\[
	g(a,b,c) > 0
\]
\[
	g(a,e,f) > 0
\]
\[
	g(b,d,f) > 0
\]
\[
	g(c,d,e) > 0
\]

Jejich součet pak můžeme upravit na:

\begin{gather*}
	g(a,b,c) + g(a,e,f) + g(b,d,f) + g(c,d,e) \\
	= -2 \cdot (a^4 + b^4 + c^4 + d^4 + e^4 +f^4) \\
	+ 2 \cdot (a^2 b^2 + a^2 b^2 + a^2 e^2 + a^2 f^2+b^2 c^2 + b^2 d^2 + b^2 f^2 + c^2 d^2 + c^2 e^2 + d^2 e^2 + d^2 f^2 + e^2 f^2 ) \\
	= g(d,e,f) + g(a, b, f) + g(a,c,e) + g(b,c,d)
\end{gather*}

Anžto víme, že výraz výše je nutně větší než 0, musí jeden z trojúhelníků složený z trojic $(d,e,f)$, $(a,b,f)$, $(a,c,e)$ a $(b,c,d)$ být nedegenerovaný. A protože ke každé z těchto trojic umíme přiřadit druhou trojici tvořící nedegenerovaný trojúhelník, dokázali jsme tvrzení ze zadání.

\end{document}
