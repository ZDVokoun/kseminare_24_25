\documentclass{fkssolpub}

\usepackage[czech]{babel}
\usepackage{fontspec}
\usepackage{fkssugar}
\usepackage{amsmath}

\author{Ondřej Sedláček}
\school{Gymnázium Oty Pavla} 
\series{1p}
\problem{4} 

\begin{document}

Jako první si musíme uvědomit, že pokud v určitém $n$-písmenném slově je výskyt každého z podslov \textit{UO} a \textit{UF} sudý nebo naopak lichý, pak jich dohromady bude vždy sudý počet a naopak. Z toho je tedy zřejmé, že veškerá slova, kde je celkový výskyt těchto slov lichý, nemohou být započítána. Zároveň je zřejmé, že tato podslova se nemohou v žádném z podslov překrývat.

Teď tedy budeme uvažovat všechna slova, kde celkový počet podslov \textit{UF} a \textit{UO} je nenulové $2m$ a kde se tato podslova nachází na stejných místech, jen se mění jejich poměr a jejich pořadí. Tehdy můžeme zanedbat, jaké písmena se nachází mimo tato podslova, protože pořadí a poměr těchto podslov nebude ovlivňovat zbylá písmena. Proto jediné, co stačí, je spočítat počet permutací těchto podslov pro každý jejich poměr a pak z nich zjistit počet slov s daným rozmístění podslov, kde každý z těchto podslov se vyskytuje suděkrát nebo lišekrát.

Počet slov s daným rozmístěním podslov, kde každý z podslov se vyskytuje suděkrát, je:

\[
	\sum_{i = 0}^{m} \frac{(2m - 2i + 2i)!}{2i! \cdot (2m - 2i)!} = \sum_{i = 0}^{m} \binom{2m}{2i}
\]

A počet podslov, kde se vyskytují lišekrát, je:

\[
	\sum_{i = 0}^{m - 1} \frac{(2m - (2i + 1) + 2i + 1)!}{(2i + 1)! \cdot (2m - (2i + 1))!} = \sum_{i = 0}^{m - 1} \binom{2m}{2i + 1}
\]

Když na tyto sumy budeme opakovaně aplikovat rekurentní vztah $\binom{n - 1}{k - 1} + \binom{n - 1}{k} = \binom{n}{k}$, zjistíme, že obě tyto sumy se rovnají výrazu:

\[
	\sum_{i = 0}^{2m - 1} \binom{2m - 1}{i}
\]

To znamená, že když si zvolíme taková slova, kde se podslova \textit{UF} a \textit{UO} nachází na stejných místech a mění se jen pořadí těchto podslov a jejich poměr, pak z nich je počet slov, kde se každý z podslov vyskytuje suděkrát, stejný jako počet slov, kde se každý z podslov vyskytuje lišekrát. Což ale zároveň znamená, že toto znamená pro všechna rozmístění podslov, tudíž pro všechna slova, kde je výskyt těchto podslov nenulový, je počet slov, kde se každý z podslov vyskytuje suděkrát, stejný jako počet slov, kde se každý z podslov vyskytuje lišekrát.

Teď když toto víme, musíme uvážit i taková slova, kde se tato podslova nenachází ani jednou. Těchto slov určitě existuje nenulové množství a řadí se mezi slova, kde se každý z podslov vyskytuje suděkrát. A to s předchozím poznatkem znamená, že počet slov, kde se každý z podslov vyskytuje suděkrát, je ostře větší než počet slov, kde se každý z podslov vyskytuje lišekrát, což jsme chtěli dokázat. Q. E. D.

\end{document}
