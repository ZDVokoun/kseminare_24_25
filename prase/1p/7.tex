\documentclass{fkssolpub}

\usepackage[czech]{babel}
\usepackage{fontspec}
\usepackage{fkssugar}
\usepackage{amsmath}

\author{Ondřej Sedláček}
\school{Gymnázium Oty Pavla} 
\series{1p}
\problem{7} 

\begin{document}

Jako první upravíme polynom na levé straně rovnice na součin:

\[
	x^3 + x^2 + x + 1 = p^n
\]
\[
	(x^2 + 1)(x + 1) = p^n
\]

Díky podmínkám ze zadání víme, že na pravé straně rovnice je přirozené číslo větší nebo rovno dvěma. Tím pádem $x$ nemůže být ani záporné číslo, ani 0, proto $x$ je nutně přirozené číslo. Díky tomu víme, že jak výraz $x^2 + 1$, tak výraz $x + 1$, je větší než jedna, proto oba tyto výrazy jsou dělitelné prvočíslem $p$. To můžeme formulovat jako soustavu kongruencí:

\[
	x^2 + 1 \equiv 0 \pmod{p}
\]
\[
	x + 1 \equiv 0 \pmod{p}
\]

Z této soustavy víme, že platí:

\[
	x^2 + 1 \equiv x + 1 \equiv 0 \pmod{p}
\]
\[
	x^2 \equiv x \pmod{p}
\]
\[
	x (x - 1) \equiv 0 \pmod{p}
\]

Z toho máme dvě řešení $x \equiv 0 \pmod{p}$ a $x \equiv 1 \pmod{p}$. Po dosazení je však zřejmé, že může platit jen řešení $x \equiv 1 \pmod{p}$, a to tehdy, když $p = 2$. Z toho plyne, že $p = 2$ a že číslo $x$ je liché číslo.

Z toho tedy víme, že výrazy $x^2 + 1$ a $x + 1$ musí být mocniny dvojky větší než jedna. Protože víme, že $x$ je liché číslo, provedeme substituci $x = 2y + 1$ a dosadíme ho do výrazu $x^2 + 1$:

\[
	x^2 + 1 = 2^a
\]
\[
	(2y + 1)^2 + 1 = 2^a
\]
\[
	4y^2 + 4y + 2 = 2^a
\]
\[
	2y^2 + 2y + 1 = 2^{a - 1}
\]

Jak můžeme vidět, ve výsledné rovnici máme na levé straně liché číslo. A protože jediná lichá celočíselná mocnina dvojky je 1, musí $y = 0$ a tedy $x = 1$.

Tohle je už dost informací, abychom získali jediné možné řešení, které je $x = 1$, $p = 2$ a $n = 2$. Protože jsme všechna ostatní řešení vyřadili, důkaz je u konce. Q.~E.~D.

\end{document}
