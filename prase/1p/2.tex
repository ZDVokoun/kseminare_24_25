\documentclass{fkssolpub}

\usepackage[czech]{babel}
\usepackage{fontspec}
\usepackage{fkssugar}
\usepackage{amsmath}

\author{Ondřej Sedláček}
\school{Gymnázium Oty Pavla} 
\series{1p}
\problem{2} 

\begin{document}

Víme, že čísel od 2 do 4321 je celkem 4320, což je číslo dělitelné čtyřmi. Zároveň víme, že výsledek tohoto výrazu musí teda být přirozené číslo. Tím pádem, když budeme schopni doplnit plusy a minusy tak, aby se výraz od 2 do 4321 vynuloval, dostaneme nejmenší možné kladné číslo, které můžeme.

To uděláme následovně -- před 2 a 4321 dáme +, před 3 a 4320 dáme -, před 4 a 4319 dáme + atd. Součet každé z těchto dvojic bude 4323 a díky tomu, že počet těchto čísel je dělitelný čtyřmi, můžeme každou přičítanou dvojici spárovat s odečítanou dvojicí. Po tomto budeme mít doplněné všechny operátory a vyjde nám, že hodnota výrazu je 1, což je nejmenší možný kladný výraz. Q. E. D.

\end{document}
