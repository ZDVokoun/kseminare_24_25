\documentclass{fkssolpub}

\usepackage[czech]{babel}
\usepackage{fontspec}
\usepackage{fkssugar}
\usepackage{amsmath}

\author{Ondřej Sedláček}
\school{Gymnázium Oty Pavla} 
\series{1p}
\problem{5} 

\begin{document}

Jako první, co uděláme, je to, že spočítáme odděleně nejmenší počet Marťanů s hlavní barvou řádku a nejmenší počet Marťanů s hlavní barvou sloupce. To nám stačí jednoduše vyjít z podmínek ze zadání; nejmenší počet Marťanů s hlavní barvou řádku je $(2n + 1) \cdot (m + 1)$, protože v každém řádku je alespoň nadpoloviční většina takových Marťanů, a s hlavní barvou sloupce $(2m + 1) \cdot (n + 1)$.

Teď si musíme uvědomit následující věc: když v součtu Marťanů s nějakou hlavní barvou bude počítán dvakrát, pak tento Marťan musel nutně mít hlavní barvu řádku i sloupce. Víme, že celkový počet Marťanů je $(2m + 1) \cdot (2n + 1)$, což je nutně taky horní odhad Marťanů, kteří mají alespoň jednu hlavní barvu. Proto, když uděláme rozdíl součtu Marťanů s nějakou hlavní barvou a horního odhadu Marťanů s nějakou barvou, dostaneme nejmenší počet Marťanů s hlavní barvou řádku i sloupce. Tento rozdíl nám vyjde:

\[
	(2n + 1)(m + 1) + (2m + 1)(n + 1) - (2m + 1)(2n + 1)\]
\[
	= 2mn + 2n + m + 1 + 2mn + 2m + n + 1 - (4mn + 2m + 2n + 1)
\]
\[
	= m + n + 1
\]

Tento výsledek je přesně ten, ke kterému jsme měli dojít. Tím pádem tvrzení ze zadání je dokázáno. Q.~E.~D.

\end{document}
