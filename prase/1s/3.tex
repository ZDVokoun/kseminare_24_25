\documentclass{fkssolpub}

\usepackage[czech]{babel}
\usepackage{fontspec}
\usepackage{fkssugar}
\usepackage{amsmath}
\usepackage{graphicx}

\author{Ondřej Sedláček}
\school{Gymnázium Oty Pavla} 
\series{1s}
\problem{3} 

\begin{document}

Nechť $x = \frac{1}{r}$. Poněvadž víme, že $x \neq 0$ a $r \neq 0$, můžeme s rovnicí, která pro $r$ platí, provést následující úpravy:

\[
	P\left(\frac{1}{r}\right) = r^2
\]

\[
	P\left(x\right) = x^{-2}
\]
\[
	P\left(x\right) - x^{-2} = 0
\]
\[
	x^2 P(x) - 1 = 0
\]

Označme polynom, který nám tady vznikl, $Q(x) = x^2 P(x) - 1$. Víme pak, že $\deg(Q) = 2026$, že vedoucí koeficient tohoto polynomu je 1 a že pro všechna celá $n$ takové, že $1 \leq |n| \leq 1012$ jsou všechna $\frac{1}{n}$ kořeny tohoto polynomu. Tím pádem můžeme najít nejvýše další dva kořeny.

U polynomu $Q$ známe kromě vedoucího koeficientu ještě dva další, a to předposlední a poslední, tím pádem z Vi\` etových vzorců můžeme získat soustavu, která určí veškeré zbývající kořeny k nalezení. Pokud řešení této soustavy budou reálná čísla, našli jsme další reálná čísla splňující vztah ze zadání. Pokud ale řešení budou komplexní, další taková čísla neexistují.

Vztah pro předposlední koeficient bude mít 2026 členů. Avšak teď uvážíme všechny členy, kde v jejich součinu jsou oba zbývající kořeny $x_1$ a $x_2$. Tehdy ten součin, kde je vynechané číslo $\frac{1}{n}$, lze vždy spárovat se součinem, kde je vynechané $ - \frac{1}{n}$ a tím pádem se vyruší. Proto celý ten součet nám stačí vyjádřit takto:

\[
	\frac{1}{1} \cdot \left(- \frac{1}{1}\right) \cdot \frac{1}{2} \cdots \left(- \frac{1}{1012}\right) \cdot \left(x_1 + x_2\right) = 0
\]
\[
	x_2 = - x_1
\]

Teď můžeme rovnou vyjádřit poslední vztah a z něj získat výsledné kořeny:

\[
	\frac{1}{1} \cdot \left(- \frac{1}{1}\right) \cdot \frac{1}{2} \cdots \left(- \frac{1}{1012}\right) \cdot x_1 \cdot x_2 = -1
\]
\[
	\left(\frac{1}{1012!}\right)^2 \cdot (- x_1^2) = -1
\]
\[
	x_1^2 = (1012!)^2
\]

Z toho tedy máme, že zbývající kořeny polynomu $Q$ jsou $x_1 = - 1012!$ a $x_2 = 1012!$. Všechna další reálná čísla splňující tento vztah jsou tedy $r = \pm \frac{1}{1012!}$.

\end{document}
