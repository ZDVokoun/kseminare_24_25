\documentclass{fkssolpub}

\usepackage[czech]{babel}
\usepackage{fontspec}
\usepackage{fkssugar}
\usepackage{amsmath}
\usepackage{graphicx}

\author{Ondřej Sedláček}
\school{Gymnázium Oty Pavla} 
\series{1s}
\problem{1} 

\begin{document}

Nejprve formulujeme Vi\` etovi vztahy pro zadaný polynom:

\[
	x_1 + x_2 + x_3 + x_4 = 0
\]
\[
	x_1 x_2 + x_1 x_3 + x_1 x_4 + x_2 x_3 + x_2 x_4 + x_3 x_4 = -40
\]
\[
	x_1 x_2 x_3 + x_1 x_3 x_4 + x_1 x_2 x_4 + x_2 x_3 x_4 = 0
\]
\[
	x_1 x_2 x_3 x_4 = q
\]

Protože kořeny $x_1$, $x_2$, $x_3$, $x_4$ tvoří aritmetickou posloupnost s diferencí $d$, pak můžeme všechny kořeny vyjádřit pomocí $x_1$ a $d$ takto:

\[
	x_2 = x_1 + d \qquad x_3 = x_1 + 2d \qquad x_4 = x_1 + 3d
\]

Když tedy dosadíme do prvního Vi\`etova vzorce, dostaneme:

\[
	4x_1 + 6d = 0 \ztoho x_1 = -\frac{3}{2} d
\]

Tedy všechny kořeny umíme vyjádřit jen pomocí diference takto:

\[
	x_1 = - \frac{3}{2} d \qquad x_2 = - \frac{d}{2} \qquad x_3 = \frac{d}{2} \qquad x_4 = \frac{3}{2} d
\]

Při dosazování do třetího Vi\` etova vztahu si všimneme toho, že první a čtvrtý člen se vyruší a že druhý a třetí se též vyruší, tudíž jakmile najdeme řešení druhého Vi\` e tova vztahu, umíme už jednoznačně určit množinu čísel $q$. Když tedy dosadíme výsledky výše do druhého Vi\` etova vztahu, spousta členů se zase vyruší a tedy dostaneme jen:

\[
	-\frac{9}{4} d^2 - \frac{1}{4} d^2 = -40 \ztoho d^2 = 16
\]

Vychází nám dvě hodnoty $d$, ale protože množina kořenů bude stejná u obou řešení, vystačíme si s $d = 4$. Teď nám zbývá dosadit jen do posledního vztahu:

\[
	q = \left(\frac{3}{2} \cdot \frac{1}{2}\right)^2 \cdot 4^4 = 4^2 \cdot 3^2 = 144
\]

Nalezli jsme tedy jediné $q$, které splňuje všechny vztahy.

\end{document}
