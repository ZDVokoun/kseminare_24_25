\documentclass{fkssolpub}

\usepackage[czech]{babel}
\usepackage{fontspec}
\usepackage{fkssugar}
\usepackage{amsmath}

\author{Ondřej Sedláček}
\school{Gymnázium Oty Pavla} 
\series{1}
\problem{1} 

\begin{document}

Protože platí z Pythagorovy věty, že přepona je nutně větší než obě z přepon, stačí nám vyvrátit jen případy, kdy odvěsna dělí jinou odvěsnu a kdy odvěsna dělí přeponu.

BÚNO předpokládejme, že $a | b$. Pak $b = ka$, kde $k \in \mathbb{N}$. Po dosazení do Pythagorovy věty dostaneme:

\[
	a^2 + k^2 a^2 = c^2
\]
\[
	(k^2 + 1) a^2 = c^2
\]

Aby $c$ bylo celé číslo, musí pak výraz $k^2 + 1$ být čtverec, což však nastane jen když $k = 0$ a tedy když $b = 0$, protože rozdíl mezi dvěma sousedními čtverci je $(x + 1)^2 - x^2 = 2x + 1$. Tento případ tedy nastat nemůže.

Teď předpokládejme, že $a | c$. Pak platí:

\[
	a^2 + b^2 = k^2 a^2
\]
\[
	b^2 = (k^2 - 1) a^2
\]

Tato rovnice nelze splnit ze stejných důvodů jako předchozí, proto i tento případ nelze splnit.

Tím jsme rozebrali veškeré případy, tedy důkaz je u konce.

\end{document}
