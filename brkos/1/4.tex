\documentclass{fkssolpub}

\usepackage[czech]{babel}
\usepackage{fontspec}
\usepackage{fkssugar}
\usepackage{amsmath}

\author{Ondřej Sedláček}
\school{Gymnázium Oty Pavla} 
\series{1}
\problem{4} 

\begin{document}

Můžeme si všimnout, že tato funkce bude nejspíš růst exponenciálně, tím pádem budeme předpokládat, že $f(n) = 2^{n - 1} + g(n)$, tudíž $g(1) = 0$ a pro $g(n)$ platí:

\[
	2^{n - 1} + g(n) = 2^{n - 1} + 2 g(n - 1) + n - 3
\]
\[
	g(n) = 2 g(n - 1) + n - 3
\]

Teď zkusíme dosadit prvních několik hodnot:

\[
	g(2) = -1 \qquad g(3) = -2 \qquad g(4) = -3
\]

Vypadá to, že funkce $g(n) = 1 - n$. To pro první členy platí, tím pádem nám zbývá dokázat indukční krok:

\[
	2 g(n - 1) + n - 3 = 2 (1 - n + 1) + n - 3 = 1 - n = g(n)
\]

Indukční krok tedy platí, tedy jsme dokázali, že $g(n) = 1 - n$, tím pádem jsme i dokázali, že $f(n) = 2^{n - 1} - n + 1$, protože rekurentní vztah pro $g(n)$ jsme odvodili ekvivalentními úpravami ze vztahu pro $f(n)$. Tím je důkaz u konce.

\end{document}
