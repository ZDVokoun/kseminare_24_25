\documentclass{fkssolpub}

\usepackage[czech]{babel}
\usepackage{fontspec}
\usepackage{fkssugar}
\usepackage{amsmath}

\author{Ondřej Sedláček}
\school{Gymnázium Oty Pavla} 
\series{1}
\problem{B} 

\begin{document}

Jako první si musíme všimnout, že v $k$-tém řádku můžeme vyjádřit součet prvních $k - 1$ čísel jako součet dvou posloupností od 1 po $k - 1$ (kromě $k = 1$). Tím pádem součet v každém řádku kromě prvního je roven součtu součtů posloupnosti od 1 do $n$ a od 1 po $k$. Celkový součet je tedy:

\[
	S = n \cdot \frac{1}{2} n (n + 1) + \sum_{k = 2}^{n} \frac{1}{2} (k - 1) k
\]

Pro výpočet sumy využijeme toho, že každý sčítanec té sumy lze vyjádřit jako kombinační číslo, které můžeme snadno sečíst:

\[
	\sum_{k = 2}^{n} \frac{1}{2} (k - 1) k = \sum_{k = 2}^{n} \binom{k}{2} = \binom{3}{3} + \sum_{k = 3}^{n} \binom{k}{2} = \binom{n + 1}{3} = \frac{n (n - 1)(n - 2)}{6}
\]

Tento výraz nám pak stačí dosadit do celkového součtu:

\[
	S = \frac{n^2 (n + 1)}{2} + \frac{n (n - 1)(n - 2)}{6} = \frac{3n^3 + 3n^2 + n^3 -3n^2 + 2n}{6} = n \cdot \frac{2n^2 + 1}{3}
\]

Tím jsme nalezli celkový součet v závislosti na $n$.


\end{document}
