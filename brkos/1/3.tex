\documentclass{fkssolpub}

\usepackage[czech]{babel}
\usepackage{fontspec}
\usepackage{fkssugar}
\usepackage{amsmath}

\author{Ondřej Sedláček}
\school{Gymnázium Oty Pavla} 
\series{1}
\problem{3} 

\begin{document}

Můžeme velmi snadno ukázat, že nerovnost $\sqrt{x^2 - 1} \leq x$ pro $x > 1$ platí:

\[
	\sqrt{x^2 - 1} \leq x
\]
\[
	x^2 - 1 \leq x^2
\]
\[
	-1 \leq 0
\]

Když tedy nerovnosti $\sqrt{x^2 - 1} \leq x$ a $\sqrt{y^2 - 1} \leq y$ dosadíme do nerovnosti ze zadání, dostaneme:

\[
	\frac{x}{\sqrt{y^2 - 1}} + \frac{y}{\sqrt{x^2 - 1}} \geq \frac{x}{y} + \frac{y}{x} \geq 2
\]

U první nerovnosti víme, že platí, proto stačí dokázat jen druhou nerovnost:

\[
	(x - y)^2 \geq 0
\]
\[
	x^2 + y^2 \geq 2xy
\]
\[
	\frac{x}{y} + \frac{y}{x} \geq 2
\]

Tímto jsme dokázali tedy nerovnost ze zadání.


\end{document}
