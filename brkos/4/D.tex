\documentclass{fkssolpub}

\usepackage[czech]{babel}
\usepackage{fontspec}
\usepackage{fkssugar}
\usepackage{amsmath}
\usepackage{graphicx}

\newcommand{\dd}{\mathrm{d}}
\renewcommand{\angle}{\sphericalangle}

\author{Ondřej Sedláček}
\school{Gymnázium Oty Pavla} 
\series{4}
\problem{D} 

\begin{document}

Víme, že graf tvořící komponentu souvislosti neobsahuje cyklus právě tehdy, když počet jeho hran je o jedna menší než počet jeho vrcholů (jedná se o vlastnost stromu). Proto tedy největší počet hran, který může mezi městy vést, je $n + m - 1$. Toto nastane tehdy, pokud všechna města tvoří jeden strom a tedy i jednu komponentu souvislosti. Je tedy zřejmé, že rozdělením na více komponent souvislosti se nevyplácí, protože by tím pro každý samostatný strom se snížil počet hran o jedna.

Teď ukážu, jak můžeme takový graf sestrojit. Jako první si vybereme jakékoli město z Horních Uher (teda ze Slovenska samozřejmě) a ten pospojujeme se všemi městy Světa. Následně vezmeme jakékoli město ze Světa a spojíme ho se všemi městy Slovenska. Tehdy celý tento graf tvoří strom se $n + m - 1$ hrany.

\end{document}
