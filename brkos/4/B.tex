\documentclass{fkssolpub}

\usepackage[czech]{babel}
\usepackage{fontspec}
\usepackage{fkssugar}
\usepackage{amsmath}
\usepackage{graphicx}

\newcommand{\dd}{\mathrm{d}}
\renewcommand{\angle}{\sphericalangle}

\author{Ondřej Sedláček}
\school{Gymnázium Oty Pavla} 
\series{4}
\problem{B} 

\begin{document}

  Indukcí dokážu, že pro každé $k \in \mathbb{Z}_{0}^{+}$ je $P_{2k} = 0$ a $P_{2k + 1} = -F_{2k + 2}$, tedy pro sudá $n$ je $P_n = 0$ a pro lichá~$n$ je $P_n = -F_{n + 1}$.

Jako první vypíšu první čtyři čísla:

\[
  P_0 = 0
\]
\[
  P_1 = -1
\]
\[
  P_2 = -1 + 0 + 1 = 0
\]
\[
  P_3 = 0 -1 - 2 = -3
\]

Můžeme tedy vidět, že pro tato čísla to opravdu platí. Můžeme se tedy pustit do důkazu pro všechna $n$.

Pro $n = 2k + 1$ podle indukčního předpokladu platí, že:

\[
  P_{2k + 1} = P_{2k} + P_{2k - 1} + (-1)^{2k + 1} F_{2k + 1} = 0 - F_{2k} - F_{2k + 1} = - F_{2k + 2}
\]

A pro $n = 2k + 2$ podle indukčního předpokladu platí, že:

\[
  P_{2k + 2} = P_{2k + 1} + P_{2k} + (-1)^{2k + 2} F_{2k + 2} = - F_{2k + 2} + 0 + F_{2k + 2} = 0
\]

Důkaz indukcí je tím u konce. Q. E. D.

\end{document}
