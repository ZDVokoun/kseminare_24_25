\documentclass{fkssolpub}

\usepackage[czech]{babel}
\usepackage{fontspec}
\usepackage{fkssugar}
\usepackage{amsmath}
\usepackage{graphicx}

\author{Ondřej Sedláček}
\school{Gymnázium Oty Pavla} 
\series{2}
\problem{3} 

\begin{document}

Když výraz ze zadání budeme modulit čtyřmi, dostaneme:

\[
	x \equiv n^{2a} + m^{2b} + 4^c \equiv (n^{a})^2 + (m^{b})^2 \pmod{4}
\]

Dále si vypíšeme kvadratické zbytky:

\[
	1^2 \equiv 1 \pmod{4}
\]
\[
	2^2 \equiv 0 \pmod{4}
\]
\[
	3^2 \equiv 1 \pmod{4}
\]
\[
	4^2 \equiv 0 \pmod{4}
\]

Ale můžeme si všimnout, že aby šlo to číslo zapsat ve tvaru výše, nemůže pro něj platit $x \equiv 3 \pmod{4}$, protože kvadratické zbytky dosahují hodnoty nejvýše $1$. Proto všechny čísla $x \equiv 3 \pmod{4}$ nelze tímto tvarem vyjádřit a těchto čísel je nekonečně mnoho. Q. E. D.

\end{document}
