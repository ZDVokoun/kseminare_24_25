\documentclass{fkssolpub}

\usepackage[czech]{babel}
\usepackage{fontspec}
\usepackage{fkssugar}
\usepackage{amsmath}
\usepackage{graphicx}

\author{Ondřej Sedláček}
\school{Gymnázium Oty Pavla} 
\series{2}
\problem{2} 

\begin{document}

Pro poslední dvojčíslí $x$ bude platit:

\[
	x \equiv 15^{15^{15}} \pmod{100}
\]

Místo toho, abychom řešili rovnou tuto rovnici, budeme řešit soustavu:

\[
	x \equiv 15^{15^{15}} \pmod{25}
\]
\[
	x \equiv 15^{15^{15}} \pmod{4}
\]

První rovnici vyřešíme snadno, jelikož číslo ze zadání je dělitelné 25:

\[
	x \equiv 15^{15^{15}} \equiv 25 \cdot 5^{15^{15} - 2} \cdot 3^{15^{15}} \equiv 0 \pmod{25}
\]

Druhou rovnici vyřešíme pomocí binomické věty:

\[
	x \equiv 15^{15^{15}} \equiv (16 - 1)^{15^{15}} \equiv (-1)^{15^{15}} \equiv -1 \equiv 3 \pmod{4}
\]

Teď pomocí čínské zbytkové věty vyřešíme rovnici:

\[
	x \equiv 0 \pmod{25}
\]
\[
	x \equiv 3 \pmod{4}
\]

Lze tedy vidět, že řešením je $x \equiv 75 \pmod{100}$, tudíž poslední dvojčíslí je 75.

\end{document}
