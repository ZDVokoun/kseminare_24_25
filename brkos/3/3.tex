\documentclass{fkssolpub}

\usepackage[czech]{babel}
\usepackage{fontspec}
\usepackage{fkssugar}
\usepackage{amsmath}
\usepackage{graphicx}

\newcommand{\dd}{\mathrm{d}}
\renewcommand{\angle}{\sphericalangle}

\author{Ondřej Sedláček}
\school{Gymnázium Oty Pavla} 
\series{3}
\problem{3} 

\begin{document}

Víme, že pro $n > 2$ platí po useknutí:

\[
  n - 2 + n + 2 - s(n) \equiv n \pmod{9}
\]

A pro $n > 11$ platí:

\[
  n + 2 - 11 \equiv n - 9 \equiv n \pmod{9}
\]

A tedy počet hlav modulo 9 se při useknutí nikdy nezmění. A protože $2 \equiv 11 \pmod{9}$, dokážeme zabít draka jen pokud $n \equiv 2 \pmod{9}$. Pro draka se 2 a 11 hlavami je postup zřejmý, pak pro každý větší $n$ budeme akorát usekávat 11 hlav, dokud jich drak nebude mít 11.

\end{document}
