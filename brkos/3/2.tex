\documentclass{fkssolpub}

\usepackage[czech]{babel}
\usepackage{fontspec}
\usepackage{fkssugar}
\usepackage{amsmath}
\usepackage{graphicx}

\newcommand{\dd}{\mathrm{d}}
\renewcommand{\angle}{\sphericalangle}

\author{Ondřej Sedláček}
\school{Gymnázium Oty Pavla} 
\series{3}
\problem{2} 

\begin{document}

Víme, že v aby šel strom jednou ukončit, musí být počet větví dělitelný třemi. Víme, že pokud počet větví v jednom stavu je $y + 2v + 3e$, kde $y$ je počet větví, na které dáme Y, $2v$ je počet větví, na které dáme V, a kde $3e$ je počet větví, na které dáme E, pak počet větví v dalším patře bude $2y + v$. Budeme chtít tedy rozhodnout, zda můžeme přidat písmena takovým způsobem, že získáme počet větví dělitelný třemi.

Přepokládejme, že to jde, a tedy $3 \mid 2y + v$. Pro počet větví ve stavu před přidáním patra platí:

\[
  y + 2v + 3e \equiv y + 2v \equiv x \pmod{3}
\]

kde $x$ je nenulový zbytek po dělení třemi. Z předpokladu pak platí:

\[
  y + 2v \equiv 2y + v + v - y \equiv v - y \equiv x \pmod{3}
\]

Což ale je ve sporu s předpokladem níže, jak uvidíme po dosazení do našeho předpokladu:

\[
  2y + v \equiv 3y + v - y \equiv v - y \equiv x \pmod{3}
\]

Tím jsme došli ke sporu a tedy strom nemůžeme nikdy ukončit.

\end{document}
