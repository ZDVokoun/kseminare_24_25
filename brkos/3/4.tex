\documentclass{fkssolpub}

\usepackage[czech]{babel}
\usepackage{fontspec}
\usepackage{fkssugar}
\usepackage{amsmath}
\usepackage{graphicx}

\newcommand{\dd}{\mathrm{d}}
\renewcommand{\angle}{\sphericalangle}
\newcommand{\lcm}{\text{lcm}}

\author{Ondřej Sedláček}
\school{Gymnázium Oty Pavla} 
\series{3}
\problem{4} 

\begin{document}

Pokud ukážeme, že pro každý pár $a,b$ platí, že $a + b \leq \lcm (a,b) + \gcd(a,b)$, součet se nikdy nezmenší. A protože původní součet je $\frac{2025 \cdot 2024}{2} > 2 \cdot 10^6$, bude nutně platit podmínka ze zadání.

Protože $a b = \lcm (a,b) \cdot \gcd(a,b)$, po dosazení dostaneme:

\[
  a + b \leq \frac{a b}{\gcd (a,b)} + \gcd(a,b)
\]

Když provedeme substituci $x = \gcd(a,b)$, postupnými úpravami dostaneme:

\[
  a + b \leq \frac{ab}{x} + x
\]
\[
  x (a + b) \leq ab + x^2
\]
\[
  x^2 - (a + b) x + ab \geq 0
\]
\[
  (x - a)(x - b) \geq 0
\]

Tato rovnice nebude platit jedině tehdy, když $\min(a,b) < \gcd(a,b) < \max(a,b)$, což ale nikdy nemůže nastat, protože $\gcd(a,b) \leq \min(a,b)$ (to vyplývá ze samotné definice nejmenšího společného dělitele). Tím jsme dokázali, že tato nerovnice vždy platí a tedy i tvrzení ze zadání.

\end{document}
