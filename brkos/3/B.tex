\documentclass{fkssolpub}

\usepackage[czech]{babel}
\usepackage{fontspec}
\usepackage{fkssugar}
\usepackage{amsmath}
\usepackage{graphicx}

\newcommand{\dd}{\mathrm{d}}
\renewcommand{\angle}{\sphericalangle}

\author{Ondřej Sedláček}
\school{Gymnázium Oty Pavla} 
\series{3}
\problem{B} 

\begin{document}

Rovnice $ax^2 + 3x + c (1 - c) = 0$ bude mít právě jedno řešení právě tehdy, když diskriminant této rovnice bude nulový:

\[
  3^2 - 4 \cdot a c (1 - c) = 0
\]

Získáme z toho rovnici, kde musíme zjistit, zda existuje řešení $c$:

\[
  9 = 4 a c (1 - c)
\]
\[
  \frac{9}{4a} = c - c^2
\]
\[
  c^2 - c + \frac{9}{4a} = 0
\]

Tato rovnice má řešení v reálných číslech právě tehdy, když diskriminant je nezáporný:

\[
  1 - 4 \cdot \frac{9}{4a} \geq 0
\]
\[
  1 \geq \frac{9}{a}
\]

Toto zřejmě platí pro všechna záporná $a$ a pro $a \geq 9$, ve zbylých intervalech rovnice neplatí nebo je pravá strana nedefinována. Proto pro $a \in (- \infty; 0) \cup \langle9; \infty)$ řešení $c$ existuje a pro zbytek nikoli.


\end{document}
