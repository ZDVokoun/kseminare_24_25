\documentclass{fkssolpub}

\usepackage[czech]{babel}
\usepackage{fontspec}
\usepackage{fkssugar}
\usepackage{amsmath}
\usepackage{graphicx}

\newcommand{\dd}{\mathrm{d}}
\newcommand{\N}{\mathbb{N}}
\newcommand{\R}{\mathbb{R}}
\newcommand{\Z}{\mathbb{Z}}
\newcommand{\Q}{\mathbb{Q}}
\renewcommand{\angle}{\sphericalangle}

\author{Ondřej Sedláček}
\school{Gymnázium Oty Pavla} 
\series{5}
\problem{A} 

\begin{document}

Pro pohodlí budu počítat s úhlem $\alpha$ zadaný v radiánech.

Délku oblouku menšího z kružnic vypočítáme takto:

\[
  a = 2 \pi r_1 - \alpha r_1
\]

Délku oblouku většího z kružnic získáme takto:

\[
  a = \alpha (r_1 + a)
\]

Protože oba tyto oblouky jsou stejně dlouhé, spojením rovnou dostaneme rovnici, která určí hledaný poměr:

\[
  2 \pi r_1 - \alpha r_1 = \alpha (r_1 + a)
\]
\[
  2 \pi r_1 - 2 \alpha r_1 = \alpha a
\]
\[
  \frac{a}{r_1} = 2 \frac{\pi - \alpha}{\alpha} = \frac{2 \pi}{\alpha} - 2
\]

To je tedy ten hledaný poměr.

\end{document}
