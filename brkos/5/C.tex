\documentclass{fkssolpub}

\usepackage[czech]{babel}
\usepackage{fontspec}
\usepackage{fkssugar}
\usepackage{amsmath}
\usepackage{graphicx}

\newcommand{\dd}{\mathrm{d}}
\newcommand{\N}{\mathbb{N}}
\newcommand{\R}{\mathbb{R}}
\newcommand{\Z}{\mathbb{Z}}
\newcommand{\Q}{\mathbb{Q}}
\renewcommand{\angle}{\sphericalangle}

\author{Ondřej Sedláček}
\school{Gymnázium Oty Pavla} 
\series{5}
\problem{C} 

\begin{document}

Jsou tři případy, kdy Járo Švrkošík získá přesně 16 bodů:

\begin{enumerate}
  \item Zvolí 8 správných odpovědí a 8 špatných
  \item Zvolí 7 správných, 5 špatných a 4 neplatných
  \item Zvolí 6 správných, 2 špatných a 8 neplatných
\end{enumerate}

První případ je snadný -- vybereme osm otázek ze šestnácti, který odpovíme správně, a u zbytku máme u otázky na výběr ze tří špatných odpovědí, proto počet kombinací v tomto případě je:

\[
  \binom{16}{8} \cdot 1^8 \cdot \binom{8}{8} \cdot 3^8
\]

V druhém případě provedeme podobný postup, jen musíme rozebrat případy, kdy se otázky zneplatní. V tomto případě se otázky zneplatní, když u dvou otázek zvolíme dvě odpovědi nebo když u jedné otázky zaškrtneme všechno. To odpovídá výrazu:

\[
  \binom{16}{7} \cdot 1^7 \cdot \binom{9}{5} \cdot 3^5 \cdot \left( \binom{4}{2} \cdot \binom{4}{2}^2 + \binom{4}{1} \cdot \binom{4}{4} \right)
\]

V třetím případě se zneplatňování odpovědí zesložití. Buď vše zaškrtáme u dvou odpovědí, u třech odpovědí vždy vybereme alespoň dvě odpovědi, nebo u čtyřech odpovědí zvolíme dvě odpovědi:

\[
  \binom{16}{6} \cdot 1^6 \cdot \binom{10}{2} \cdot 3^2 \cdot \left( \binom{8}{2} \cdot \binom{4}{4}^2 + \binom{8}{3} \cdot \binom{4}{2}^3 \cdot \binom{6}{2} + \binom{8}{4} \cdot \binom{4}{2}^4 \right)
\]

A protože celkový počet kombinací je $\binom{64}{16}$, celková pravděpodobnost je:

\[
  p = \frac{1}{\binom{64}{16}} \cdot \left(
    \binom{16}{8} \cdot 3^8 + 
    \binom{16}{7} \cdot \binom{9}{5} \cdot 3^5 \cdot \left(\binom{4}{2}^3 + 4 \right) +  \
    \binom{16}{6} \cdot \binom{10}{2} \cdot 3^2 \cdot \left( \binom{8}{2} + \binom{8}{3} \cdot \binom{4}{2}^3 \cdot \binom{6}{2} + \binom{8}{4} \cdot \binom{4}{2}^4 \right)
  \right)
\]
\[
  = \frac{10665720351}{5428077078662} \doteq 0{,}196492 \%
\]

\end{document}
