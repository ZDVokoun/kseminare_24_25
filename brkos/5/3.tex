\documentclass{fkssolpub}

\usepackage[czech]{babel}
\usepackage{fontspec}
\usepackage{fkssugar}
\usepackage{amsmath}
\usepackage{graphicx}

\newcommand{\dd}{\mathrm{d}}
\renewcommand{\angle}{\sphericalangle}

\author{Ondřej Sedláček}
\school{Gymnázium Oty Pavla} 
\series{5}
\problem{3} 

\begin{document}

Víme tedy, že tento polynom můžeme vyjádřit jako:

\[
  x^6 + b x^5 + c x^4 - c x^2 - b x - 1
\]

Můžeme si ale všimnout, že tento polynom má kořeny 1, $-1$, proto lze tento polynom rozložit na:

\[
  {\left(x^{4} + b x^{3} + (c + 1) x^{2} + b x + 1\right)} {\left(x + 1\right)} {\left(x - 1\right)}
\]

Tudíž jsme dostali další reciproký polynom, ale čtvrtého stupně. Nechť je tento nový polynom $Q(x)$. Zároveň pokud by kořen $Q(x)$ byla 1, musela by zároveň i $-1$ být kořenem $Q(x)$, aby se zachovala parita reciprokých kořenů, a naopak. Tehdy ale součet dvojnásobných kořenů by byl 0, tím pádem ani 1, ani $-1$, nemohou být dvojnásobnými kořeny. Tedy $Q(x)$ má oba dvojnásobné kořeny, které má zadaný polynom, tím pádem těmi dvojnásobnými kořeny musí být čísla $a$, $1/a$ pro nějaké $a$, které můžeme snadno najít:

\[
  a + \frac{1}{a} = -4
\]
\[
  a^2 + 1 = -4a
\]
\[
  a^2 + 4a + 1 = 0
\]
\[
  a = \pm \sqrt{3} - 2
\]

Obě řešení $a$ jsou si navzájem reciproké, tedy hledaný polynom umíme vyjádřit jako:

\[
  (x - (- \sqrt{3} - 2))^2 (x - (\sqrt{3} - 2))^2 (x + 1) (x - 1) = x^{6} + 8 \, x^{5} + 17 \, x^{4} - 17 \, x^{2} - 8 \, x - 1
\]

Tento polynom splňuje všechny podmínky ze zadání, tedy se jedná opravdu o něj. Našli jsme tedy hledaný polynom.

\end{document}
