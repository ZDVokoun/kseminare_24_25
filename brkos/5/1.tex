\documentclass{fkssolpub}

\usepackage[czech]{babel}
\usepackage{fontspec}
\usepackage{fkssugar}
\usepackage{amsmath}
\usepackage{graphicx}

\newcommand{\dd}{\mathrm{d}}
\renewcommand{\angle}{\sphericalangle}
\newcommand{\N}{\mathbb{N}}

\author{Ondřej Sedláček}
\school{Gymnázium Oty Pavla} 
\series{5}
\problem{1} 

\begin{document}

Budu se dívat jen na podmnožinu polynomů, které jde z těchto čísel složit, a to na $x^2 + b x + c$. Kořeny tohoto polynomu jsou:

\[
  x = \frac{-b \pm \sqrt{b^2 - 4c}}{2}
\]

Tedy dostaneme reálný kořen právě tehdy, když:

\[
  b^2 > 4c
\]

Teď si zavedu proměnné $x$ a $y$, kde $x = \max(a,b)$ a $y = \min(a,b)$. Nutně platí, že $y \geq 2$ a že $x \geq y + 1$.

Teď chci dokázat následující nerovnost:

\[
  (y + 1)^2 > 4y
\]
\[
  y^2 + 2y + 1 > 4y
\]
\[
  y^2 - 2y + 1 > 0
\]
\[
  (y - 1)^2 > 0
\]

Tato nerovnost díky podmínce $y \geq 2$ nutně platí. A protože pracujeme v přirozených číslech, kvadratická funkce je rostoucí a platí:

\[
  x^2 \geq (y + 1)^2 > 4y
\]

Tedy když přepermutujeme koeficienty tak, aby $b = x$ a $c = y$, dostaneme polynom, který chceme.

\end{document}
