\documentclass{fkssolpub}

\usepackage[czech]{babel}
\usepackage{fontspec}
\usepackage{fkssugar}
\usepackage{amsmath}
\usepackage{graphicx}

\newcommand{\dd}{\mathrm{d}}
\newcommand{\N}{\mathbb{N}}
\renewcommand{\angle}{\sphericalangle}

\author{Ondřej Sedláček}
\school{Gymnázium Oty Pavla} 
\series{5}
\problem{4} 

\begin{document}

\section{Na nulový polynom}

Jako první se podíváme na polynom prvního stupně, a to na $- x - 1$. Můžeme si všimnout toho, že tento polynom má kořen $-1$ a všechny jeho koeficienty jsou $-1$, tedy se zobrazí na nulový polynom. 

Teď ukážu, že abychom dostali takový polynom, ale vyššího stupně, stačí nám vynásobit tento polynom nějakou mocninou $x^n$. Protože kořeny polynomu $x^n (-x - 1) = -x^{n + 1} - x^n$ jsou vždy čísla 0 a $-1$ a taky tento polynom nemá jiné koeficienty kromě těchto dvou, ukázali jsme, že polynom, který se zobrazí na nulový polynom, existuje pro všechny stupně.

\section{Sám na sebe}

Víme, že pro všechna $n \in \N$ platí, že $1^n = 1$ a $0^n = 0$. A protože polynom $x^n$ má jen koeficienty 0 a 1, musí se nutně zobrazit sám na sebe.

\section{$k$-násobek}

Ukáži, že všechny polynomy ve tvaru $\sqrt[n]{k} x^n$ se převedou na $k$-násobek původního polynomu:

\[
  F(\sqrt[n]{k} x^n) = \left(\sqrt[n]{k} \left(\sqrt[n]{k}\right)^n\right) x^n = k \cdot \sqrt[n]{k} x^n
\]

Tento polynom existuje pro každý stupeň, proto tedy pro každý stupeň existuje polynom, které se zobrazí na jakýkoli přirozený násobek.

\end{document}
