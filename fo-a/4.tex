\documentclass{fkssolpub}

\usepackage[czech]{babel}
\usepackage{fontspec}
\usepackage{fkssugar}
\usepackage{amsmath}
\usepackage{graphicx}
\usepackage[version=4]{mhchem}

\author{Ondřej Sedláček}
\school{Gymnázium Oty Pavla} 
\series{}
\problem{4} 

\begin{document}

V periodické tabulce najdeme, že protonové číslo polonia je $Z = 84$, tedy součet hmotností protonů a neutronů je:

\[
	m = Z \cdot m_p + (A - Z) \cdot m_n = 3{,}5154 \cdot "10^{-25} kg"
\]

Tudíž je vidět, že $m > m_{\text{Po}}$. V rozdílu hmotností je právě uložena vazebná energie, kterou určíme pomocí Einsteinova vzorce:

\[
	E = B \cdot c^2 = (m - m_{\text{Po}}) c^2 \cdot 2{,}63607 \cdot "10^{-10} J" = "1645{,}30218 MeV"
\]

Množství atomů izotopu polonia v původním vzorku najdeme následovně:

\[
	N_0 = \frac{m_0}{m_{\text{Po}} + 84m_e} = 2{,}86792 \cdot 10^{18}
\]

Rovnice reakce polonia 210 je:

\[
	\ce{^{210}_{84}Po -> ^{206}_{82}Pb + ^4_2He}
\]

Podle zákona radioaktivní přeměny platí následující rovnice:

\[
	N = N_0 2^{-\frac{t}{T}}
\]

Tu upravíme a zlogaritmujeme:

\[
	\frac{N}{N_0} = 2^{-\frac{t}{T}}
\]
\[
	\log_2 N - \log_2 N_0 = - \frac{t}{T}
\]
\[
	T = \frac{t}{\log_2 N_0 - \log_2 N} = "138{,}0017 d"
\]

Pro počáteční aktivitu vzorku pak platí:

\[
	A_0 = \lambda N_0 = N_0 \cdot \frac{\ln 2}{T} = 1{,}66723 \cdot "10^{11} Bq"
\]

Čas, kdy ve vzorku bude 10\% atomů, zjistíme podobně jako poločas rozpadu:

\[
	\frac{N_0}{10} = N_0 2^{-\frac{t}{T}}
\]
\[
	- \log_2 10 = - \frac{t}{T}
\]
\[
	t = T \log_2 10 = "458{,}43179 d"
\]


\end{document}
