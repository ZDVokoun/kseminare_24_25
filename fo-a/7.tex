\documentclass{fkssolpub}

\usepackage[czech]{babel}
\usepackage{fontspec}
\usepackage{fkssugar}
\usepackage{amsmath}
\usepackage{graphicx}

\author{Ondřej Sedláček}
\school{Gymnázium Oty Pavla} 
\series{}
\problem{7} 

\renewcommand{\d}{\mathrm{d}}

\begin{document}

Jako první vyjádříme přímku procesu, a to vůči $V_0$ a $p_0$. Víme, že proces bude vyjádřen rovnicí:

\[
	p = k x V_0 + l
\]

Z úseku $p_0$ víme, že $l = p_0$, a pak následně z úseku $V_0$ zjistíme $k = - \frac{p_0}{V_0}$. Tedy rovnice je:

\[
	p = p_0 (1 - x)
\]

Pro teplotu $T$ víme, že platí:

\[
	\frac{p_A V_A}{T_A} = \frac{p V}{T}
\]
\[
	\frac{\frac{3}{4} p_0 \cdot \frac{1}{4} V_0}{T_A} = \frac{x (1 - x) p_0 V_0}{T}
\]
\[
	3 T = 16 x (1 - x) T_A
\]
\[
	T = \frac{16 x (1 - x) T_A}{3}
\]

Dále potřebujeme zjistit látkové množství $n$:

\[
	p_A V_A = n R T_A
\]
\[
	\frac{3}{16} p_0 V_0 = n R T_A
\]
\[
	n = \frac{3 p_0 V_0}{16 R T_A}
\]

Když teď zjistíme změnu vnitřní energie plynu a práci vykonanou plynem, z prvního termodynamického zákona zjistíme teplo. Proto jako první zjistíme změnu vnitřní energie:

\[
	\Delta U(x) = n C_{Vm} (T - T_A) = \frac{3 p_0 V_0}{16 R T_A} \cdot \frac{5R}{2} \cdot \left(\frac{16 x (1 - x) T_A}{3} - T_A\right) = \frac{5 p_0 V_0 (16 x (1 - x) - 3)}{32}
\]

Práce konaná na plynu je:

\[
	W(x) = - \int_{V_A}^V p \, \d V = - \int_{1/4}^x p V_0 \, \d x= - V_0 p_0 \int_{1/4}^x (1-x) \, \d x
\]
\[
	= - V_0 p_0 \left(- \frac{1}{2} x^2 + x + \frac{1}{2} \cdot \frac{1}{16} - \frac{1}{4}\right) = \frac{V_0 p_0 \left(16 x^2 - 32 x + 7\right)}{32}
\]

A teď umíme tedy zjistit teplo, který vejde do plynu:

\[
	Q(x) = \Delta U(x) - W(x) = \frac{5 p_0 V_0 (16 x (1 - x) - 3)}{32} - \frac{V_0 p_0 \left(16 x^2 - 32 x + 7\right)}{32} = \frac{p_0 V_0 (-96 x^2 + 112x - 22)}{32}
\]

Víme, že pro $V$, $V_A \leq V \leq V_C$, bude přijaté teplo během přechodu $V_A \to V$ hodnota $Q(x)$ růst, ale naopak pro $V$, $V_C \leq V \leq V_B$ bude $Q(x)$ klesat. Proto nám stačí najít maximum funkce $Q(x)$ pomocí derivací:

\[
	(Q(x))' = 0
\]
\[
	\left( \frac{p_0 V_0 (-96 x^2 + 112x - 22)}{32} \right)' = 0
\]
\[
	-192x_C + 112 = 0
\]
\[
	x_C = \frac{7}{12}
\]

Poněvadž tento výsledek se nachází v povoleném rozmezí $\frac{1}{4} \leq x_C \leq \frac{3}{4}$, platí tedy $V_C = \frac{7}{12} V_0$.

Přijaté teplo $Q_p$ je pak:

\[
	Q_p = Q(x_C) = \frac{32}{3} \cdot \frac{p_0 V_0}{32} = \frac{p_0 V_0}{3}
\]

Pro vydané teplo $Q_v$ pak platí:

\[
	Q_v = Q_p - Q(x_B) = \frac{p_0 V_0}{3} - 8 \cdot \frac{p_0 V_0}{32} = \frac{p_0 V_0}{12}
\]

\end{document}
