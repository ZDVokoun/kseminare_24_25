\documentclass{fkssolpub}

\usepackage[czech]{babel}
\usepackage{fontspec}
\usepackage{fkssugar}
\usepackage{amsmath}
\usepackage{graphicx}

\newcommand{\dd}{\mathrm{d}}

\author{Ondřej Sedláček}
\school{Gymnázium Oty Pavla} 
\series{66-1}
\problem{2} 

\begin{document}

Hvězda je zřejmě přibližně koule, proto střední hodnota hustoty bude:

\[
	\rho = \frac{M}{V} = \frac{M}{\frac{4}{3} \pi R^3} = \frac{3M}{4 \pi R^3} = 2{,}74599 \cdot "10^{9} kg \cdot m^{-3}"
\]

Abychom zjistili frekvenci $\lambda_m$, pro kterou je spektrální hustota vyzařování maximální, použijeme Wienův posunovací zákon:

\[
	\lambda_m = \frac{b}{T} = "1{,}156 \cdot 10^{-7} m" = "115{,}6 nm"
\]

Protože známe jenom pokles průměrné teploty hvězdy, předpokládejme, že za tu dobu poklesne povrchová teplota o stejné množství.

Ze Stefanova-Boltzmannova zákona dokážeme zjistit zářivý tok hvězdy a tedy i okamžitou změnu energie při teplotě $T$:

\[
	\frac{\Phi}{S} = \frac{- \dd E}{\dd t \cdot S} = \sigma T^4
\]

Zároveň z definice měrné tepelné kapacity platí:

\[
	\dd E = M c \cdot \dd T
\]

Z toho dostaneme rovnici, kterou vyřešíme:

\[
	- \sigma T^4 S \cdot \dd t = M c \cdot \dd T
\]
\[
	- \frac{\sigma S}{M c} \, \dd t = T^{-4}  \, \dd T
\]
\[
	\int_0^{\tau } - \frac{\sigma S}{M c} \, \dd t = \int_T^{T - \Delta T} T^{-4}  \, \dd T
\]
\[
	- \frac{\sigma S \tau}{M c} = - \frac{1}{3} ((T - \Delta T)^{-3} - T^{-3})
\]
\[
	c = \frac{3 \sigma S \tau}{M ((T - \Delta T)^{-3} - T^{-3})} = \frac{12 \pi R^2 \sigma \tau}{M ((T - \Delta T)^{-3} - T^{-3})} = 5{,}410653 \cdot "10^{11} kg^{-1} \cdot K^{-1} \cdot J"
\]

Při tom, co foton bude opouštět atmosféru, bude změna energie ve vzdálenosti $h$ od hvězdy roven:

\[
	- \dd E = G \frac{m M}{(R + h)^2} \cdot \dd h
\]

Protože má foton hybnost, můžeme mu přiřadit ekvivalentní hmotnost podle vzorce $E = m c^2$, podle toho určit, jaká je energie fotonu při opuštění gravitačního pole hvězdy (předpokládáme, že vzdálenost pozorovatele se blíží nekonečnu), a následně výslednou vlnovou délku:

\[
	- \dd E = G \frac{\frac{E}{c^2} M}{(R + h)^2} \cdot \dd h
\]
\[
	\frac{1}{E} \, \dd E = - \frac{G M}{c^2 (R + h)^2} \, \dd h
\]
\[
	\int_{E'}^{E} \frac{1}{E} \, \dd E = \int_0^{+\infty} - \frac{G M}{c^2 (R + h)^2} \, \dd h
\]
\[
	[\ln E]_{E'}^{E} = - \frac{GM}{c^2} \cdot \left[\frac{1}{R + h}\right]_0^{+\infty}
\]
\[
	\ln \frac{E'}{E} = - \frac{GM}{R c^2}
\]
\[
	E' = E \cdot \exp\left(-\frac{GM}{R c^2}\right)
\]
\[
	\frac{h c}{\lambda} = \frac{hc}{\lambda_0} \cdot \exp\left(-\frac{GM}{R c^2}\right)
\]
\[
	\lambda = \lambda_0 \cdot \exp\left(\frac{GM}{R c^2}\right)
\]

Teď nám tedy zbývá zjistit relativní změnu vlnové délky:

\[
	z = \frac{\Delta \lambda}{\lambda} = \frac{\lambda - \lambda_0}{\lambda_0} = \exp\left(\frac{GM}{R c^2}\right) - 1 = 2{,}6737 \cdot 10^{-4}
\]


\end{document}
