\documentclass{fkssolpub}

\usepackage[czech]{babel}
\usepackage{fontspec}
\usepackage{fkssugar}
\usepackage{amsmath}
\usepackage{graphicx}

\author{Ondřej Sedláček}
\school{Gymnázium Oty Pavla} 
\series{}
\problem{6} 

\begin{document}

Z toho, že budeme měřit ohniskovou vzdálenost spojné čočky, víme, že budou platit následující rovnice ($a$ je předmětová a $a'$ je obrazová vzdálenost):

\[
	l = a + a'
\]
\[
	\frac{1}{f} = \frac{1}{a} + \frac{1}{a'}
\]

Kombinací těchto dvou rovnic dostaneme rovnici kvadratickou, kterou vyřešíme:

\[
	\frac{1}{f} = \frac{1}{a} + \frac{1}{l - a}
\]
\[
	a (l - a) = f l
\]
\[
	- a^2 + a l - fl = 0
\]
\[
	a_1 = \frac{l - \, \sqrt{l^{2}-4 \, f l}}{2} \qquad a_2 = \frac{l + \, \sqrt{l^{2} - 4 \, f l}}{2}
\]

Z řešení $a_1$ vidíme, že $l^2 - 4 fl \geq 0$, a tedy:

\[
	\frac{l}{f} \geq 4
\]

Obě řešení jsou jinak nutně kladná, protože $l > \sqrt{l^2 - 4 f l}$.

Pro důkaz vzorce ze zadání nejprve dosadíme námi nalezená řešení do $d$:

\[
	d = |a_1 - a_2| =  \left| \frac{l - \, \sqrt{l^{2}-4 \, f l}}{2} - \frac{l + \, \sqrt{l^{2} - 4 \, f l}}{2} \right| = \sqrt{l^{2} - 4 \, f l}
\]

Následně ji dosadíme do výrazu $\frac{l^2 - d^2}{4l}$:

\[
	\frac{l^2 - d^2}{4l} = \frac{l^2 - (l^{2} - 4 \, f l)}{4l} = \frac{4fl}{4l} = f
\]

Čímž jsme vzorec ze zadání dokázali. Tím jsme dokončili teoretickou část.

Na experiment jsem použil svíčku, černý papír jako stínítko opřený o zeď, metr a tenkou spojnou čočku. Výsledky měření jsou tedy následující:

\begin{table}[h!]
	\caption{Výsledky měření}\label{tab:1}
	\begin{center}
		\begin{tabular}{|c|c|c|c|c|c|}
			\hline
			Číslování & $l$ [m] & $d$ [m] & $f$   & $\Delta f$ & $\delta f [\%]$ \\
			\hline
			1 & 1 & 0,473 & 0,194 & 0,022 & 10,084  \\
			2 & 1,2 & 0,605 & 0,224 & 0,008 & 3,666  \\
			3 & 1,4 & 0,834 & 0,226 & 0,010 & 4,615  \\
			4 & 1,6 & 1,153 & 0,192 & 0,024 & 10,913  \\
			5 & 1,8 & 1,22 & 0,243 & 0,027 & 12,716  \\
 			$\prumer$ &  &  & 0,216 & 0,018 & 8,399  \\
			\hline
		\end{tabular}
	\end{center}
\end{table}

Naměřili jsme tedy hodnotu $f = (0{,}216 \pm "0{,}02) m"$ s relativní odchylkou $\delta f = 8{,}399 \%$. Jeden z významných faktorů, který mohl způsobit takovou odchylku, je fakt, že jsme museli určovat ostrý obraz od oka, což není zcela spolehlivé. Zároveň to mohlo být způsobeno tím, že svíčka, čočka a stínítko nemusely být nutně zarovnany správně. Výsledky jsou ale i přes tyto okolnosti uspokojující.


\end{document}
