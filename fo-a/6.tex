\documentclass{fkssolpub}

\usepackage[czech]{babel}
\usepackage{fontspec}
\usepackage{fkssugar}
\usepackage{amsmath}
\usepackage{graphicx}

\author{Ondřej Sedláček}
\school{Gymnázium Oty Pavla} 
\series{}
\problem{6} 

\begin{document}

Z toho, že budeme měřit ohniskovou vzdálenost spojné čočky, víme, že budou platit následující rovnice ($a$ je předmětová a $a'$ je obrazová vzdálenost):

\[
	l = a + a'
\]
\[
	\frac{1}{f} = \frac{1}{a} + \frac{1}{a'}
\]

Kombinací těchto dvou rovnic dostaneme rovnici kvadratickou, kterou vyřešíme:

\[
	\frac{1}{f} = \frac{1}{a} + \frac{1}{l - a}
\]
\[
	a (l - a) = f l
\]
\[
	- a^2 + a l - fl = 0
\]
\[
	a_1 = \frac{l - \, \sqrt{l^{2}-4 \, f l}}{2} \qquad a_2 = \frac{l + \, \sqrt{l^{2} - 4 \, f l}}{2}
\]

Z řešení $a_1$ vidíme, že $l^2 - 4 fl \geq 0$, a tedy:

\[
	\frac{l}{f} \geq 4
\]

Obě řešení jsou jinak nutně kladná, protože $l > \sqrt{l^2 - 4 f l}$.

Pro důkaz vzorce ze zadání nejprve dosadíme námi nalezená řešení do $d$:

\[
	d = |a_1 - a_2| =  \left| \frac{l - \, \sqrt{l^{2}-4 \, f l}}{2} - \frac{l + \, \sqrt{l^{2} - 4 \, f l}}{2} \right| = \sqrt{l^{2} - 4 \, f l}
\]

Následně ji dosadíme do výrazu $\frac{l^2 - d^2}{4l}$:

\[
	\frac{l^2 - d^2}{4l} = \frac{l^2 - (l^{2} - 4 \, f l)}{4l} = \frac{4fl}{4l} = f
\]

Čímž jsme vzorec ze zadání dokázali. Tím jsme dokončili teoretickou část.

\end{document}
