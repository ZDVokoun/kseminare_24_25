\documentclass{fkssolpub}

\usepackage[czech]{babel}
\usepackage{fontspec}
\usepackage{fkssugar}
\usepackage{amsmath}
\usepackage{graphicx}

\author{Ondřej Sedláček}
\school{Gymnázium Oty Pavla} 
\series{2}
\problem{1} 

\begin{document}

Grafické karty studenta za rok spotřebují tolik energie:

\[
	E = P \cdot t = "26280 kWh"
\]

Aby se mu těžení bitcoinu vyplatilo, musí platit nerovnost:

\[
	0{,}2 c > 1000 k + E c_z
\]

kde $c$ je cena bitcoinu, $c_z$ je cena energie a $k$ je počet pořízených stromů.

Jako první vyřešíme uhlí. Při něm jsou roční emise:

\[
	e_u = 0{,}82 E = "21549{,}6 kg"
\]

Počet stromů pak je:

\[
	k_u = \left\lceil \frac{e_u}{25} \right\rceil = 862
\]

A tedy pro cenu bitcoinu platí:

\[
	c > 5 \cdot (1000 k_u + E c_u) = "5009048 \text{Kč}"
\]

Pro vodní elektrárnu bude postup analogický:

\[
	e_v = 0{,}012 E = "315.36 kg"
\]
\[
	k_v = \left\lceil \frac{e_v}{25} \right\rceil = 13
\]
\[
	c > 5 \cdot (1000 k_v + E c_v) = "590600 \text{Kč}"
\]

Pro vodní elektrárnu je cena, při které se těžění vyplatí, až o řád menší. Tím je tedy úloha vyřešena.

\end{document}
