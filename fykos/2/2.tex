\documentclass{fkssolpub}

\usepackage[czech]{babel}
\usepackage{fontspec}
\usepackage{fkssugar}
\usepackage{amsmath}
\usepackage{graphicx}

\author{Ondřej Sedláček}
\school{Gymnázium Oty Pavla} 
\series{2}
\problem{2} 

\begin{document}

Když budeme předpokládat, že v kapiláře dochází k dokonalému smáčení, platí vztah:

\[
	h \rho g = \frac{2\sigma}{R}
\]

Tento vztah tedy upravíme:

\[
	h = \frac{2 \sigma}{\rho g \frac{d}{2}} = \frac{4 \sigma}{\rho g d} \doteq "0{,}294 m"
\]

Když se podíváme do tabulek na běžné kapaliny, žádná kapalina nemá větší poměr $\sigma/\rho$ než voda, proto je voda pro tento účel nejvhodnější.

Jeden z hlavních důvodů, proč se voda v rostlinách dostane mnohem výš, je transpirace. Ta způsobuje podtlak v listech, která díky kohezi vody způsobí to, že molekuly vody se dostanou výše. Proto se výsledek tolik liší od reálných stromů.

\end{document}
