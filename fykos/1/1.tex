\documentclass{fkssolpub}

\usepackage[czech]{babel}
\usepackage{fontspec}
\usepackage{fkssugar}
\usepackage{amsmath}

\author{Ondřej Sedláček}
\school{Gymnázium Oty Pavla} 
\series{1}
\problem{1} 

\begin{document}

Víme, že Newtonův gravitační zákon říká, že na hmotný bod o hmotnosti $m$ při povrchu Europy působí síla:

\[
	F\_{g} = G \cdot \frac{m M\_{E}}{R\_{E}^2}
\]

Z čehož platí, že:

\[
	g\_{E} = G \cdot \frac{M\_{E}}{R\_{E}^2}
\]

Protože chceme vědět, jaká bude hmotnost Europy, když bude homogenní koulí z kapalné vody nebo ledu, bude obecně platit vzorec:

\[
	M\_{E} = \frac{4}{3} \pi R\_{E}^3 \cdot \rho
\]

Po dosazení do vztahu výše dostaneme:

\[
	g\_{E} = \rho \cdot G \frac{\frac{4}{3} \pi R\_{E}^3}{R\_{E}^2} = \rho \cdot \frac{4}{3} G \pi R\_{E}
\]

Dosadíme tam tedy postupně hodnoty z matematických a fyzikálních tabulek:

\[
	g\_{E, voda} \doteq \rho\_{voda} \cdot 4,3891 \cdot "10^{-3} m \cdot s^{-2}" \doteq "0,43803 m \cdot s^{-2}"
\]
\[
	g\_{E, led} \doteq \rho\_{voda} \cdot 4,3891 \cdot "10^{-3} m \cdot s^{-2}" \doteq "0,4038 m \cdot s^{-2}"
\]

Když to porovnáme se zrychlením na Europě $"1.314 m \cdot s^2"$, zjistíme, že oboje zrychlení je skoro třikrát menší než skutečné zrychlení. U ledu se zrychlení logicky zmenšilo, protože hustota ledu je menší než hustota kapalného vzduchu.

\end{document}
