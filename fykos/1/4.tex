\documentclass{fkssolpub}

\usepackage[czech]{babel}
\usepackage{fontspec}
\usepackage{fkssugar}
\usepackage{amsmath}

\author{Ondřej Sedláček}
\school{Gymnázium Oty Pavla} 
\series{1}
\problem{4} 

\begin{document}

Víme, že pro vzdálenost předmětovou $a$, obrazovou $a'$ a ohniskovou $f$ platí vztah:

\[
	\frac{1}{a} + \frac{1}{a'} = \frac{1}{f}
\]

Protože $D = a + a'$ a obrazová vzdálenost $a'$ je vzdálenost čočky od stropu, určíme pozici čočky následovně:

\[
	\frac{1}{D - a'} + \frac{1}{a'} = \frac{1}{f}
\]
\[
	f \cdot \frac{a' + D - a'}{a' (D - a')} = 1
\]
\[
	fD = - a'^2 + Da'
\]
\[
	a'^2 - Da' + fD = 0
\]

Řešení této rovnice jsou:

\[
	a' = \frac{D \pm \sqrt{D (D - 4f)}}{2}
\]

U volného pádu víme, že vzdálenost předmětu od stropu při volném pádu v čase $t$ bude:

\[
	D_p = D + \frac{1}{2} g t^2
\]

Při dosazení $D_p$ místo $D$ v rovnici pro $a'$ získáme:

\[
	a'_p = \frac{D + \frac{1}{2} gt^2 \pm \sqrt{\left(D + \frac{1}{2} g t^2\right) \left( D + \frac{1}{2} g t^2  - 4f\right)}}{2}
\]

A poněvadž když ji necháme padat hodně dlouho tak bude platit, že $\frac{1}{2} g t^2 \gg D$ a $\frac{1}{2} g t^2 \gg f$, tak se výraz výše zjednodušší na:

\[
	a'_p = \frac{\frac{1}{2} gt^2 \pm \sqrt{\left(\frac{1}{2} g t^2\right) \left(\frac{1}{2} g t^2\right)}}{2}
\]
\[
	a'_p = \frac{\frac{1}{2} gt^2 \pm \frac{1}{2} g t^2}{2}
\]
\[
	a'_p \in \left\{0, \frac{1}{2} g t^2\right\}
\]

Tudíž postupem času se bude čočka přibližovat buď ke stropu, nebo k čočce.


\end{document}
