\documentclass{fkssolpub}

\usepackage[czech]{babel}
\usepackage{fontspec}
\usepackage{fkssugar}
\usepackage{amsmath}

\author{Ondřej Sedláček}
\school{Gymnázium Oty Pavla} 
\series{1}
\problem{5} 

\begin{document}

Protože předpokládáme, že $A \gg d^2$, je elektrické pole mezi deskami kondenzátoru homogenní. A poněvadž umíme zjistit velikost elektrické intenzity v blízkosti desky, zjistíme velikost elektrické intenzity mezi deskami takto:

\[
	E = \frac{\sigma}{2 \epsilon} = \frac{Q}{2 A \epsilon}
\]

Na desky působí dvě síly -- elektrostatická a síla pružiny. Tyto síly působí přímo proti sobě a musí být v rovnováze. Z toho získáme rovnici, ze které můžeme vyjádřit výchylku pružiny $y$ vzhledem k velikosti náboje $Q$:

\[
	E Q = k y
\]
\[
	\frac{Q^2}{2 A \epsilon} = k y
\]
\[
	y = \frac{Q^2}{2 A \epsilon k}
\]

Práce, kterou musíme vykonat k nabití kondenzátoru, se převede na energii obou pružin a na energii kondenzátoru. Proto platí:

\[
	W = 2 \cdot \frac{1}{2} k y^2 + \frac{1}{2} \cdot \frac{Q^2}{C}
	= \frac{Q^4}{4 A^2 \epsilon^2 k} + \frac{Q^2 (d - \frac{Q^2}{A \epsilon k})}{2 A \epsilon}
	= \frac{Q^4}{4 A^2 \epsilon^2 k} + \frac{Q^2 (d A \epsilon k - Q^2)}{2 A^2 \epsilon^2 k}
	= \frac{- Q^4 + 2 Q^2 d A \epsilon k}{4 A^2 \epsilon^2 k}
\]

Pro nalezení maximálního napětí nejprve vyjádříme vzorec napětí $U$ vzhledem k náboji $Q$. Vyjdu přitom z definice kapacity:

\[
	U = \frac{Q}{C} = \frac{Q (d - 2y)}{A \epsilon} = \frac{Q d A \epsilon k - Q^3}{A^2 \epsilon^2 k}
\]

Tento vzorec můžeme následně zderivovat vzhledem k $Q$:

\[
	\frac{\d U}{\d Q} = \frac{d A \epsilon k - 3 Q^2}{A^2 \epsilon^2 k}
\]

Extrémy jsou tedy v bodech, kdy:

\[
	\frac{d A \epsilon k - 3 Q^2}{A^2 \epsilon^2 k} = 0
\]
\[
	d A \epsilon k = 3 Q^2
\]
\[
	Q = \pm \sqrt{\frac{d A \epsilon k}{3}}
\]

Nalezli jsme tedy dva extrémy, z nichž jeden je pro naše účely neplatný, tedy musíme jen ukázat, že ten extrém v kladných hodnotách je maximum a že kondenzátor můžeme nabít nábojem o dané velikosti.

Fakt, že extrém v kladných hodnotách je maximum v kladných hodnotách, lze ukázat jednoduše na grafu derivace funkce $U(Q)$. Jejím grafem je zřejmě parabola ve tvaru "kopce" (má před nejvyšším členem negativní koeficient) a už jsme zjistili, že má dva různé kořeny, proto tedy nejdříve klesá, chvíli vyroste a pak zase klesá. V bodech, kdy se funkce mění z rostoucí na klesající a naopak, máme nalezené extrémy. Anžto kladný extrém je druhý v pořadí, musí tedy nutně být maximem v kladných číslech.

Teď musíme zkontrolovat, zda vůbec můžeme nabít kondenzátor tímto nábojem. Víme, že $d > 2y$, z čehož dostaneme:

\[
	d > 2 \cdot \frac{Q^2}{2 A \epsilon k} = \frac{Q^2}{A \epsilon k}
\]
\[
	\sqrt{d A \epsilon k} > Q
\]

Což když dosadíme námi nalezený náboj, kdy je napětí maximální, dostaneme:

\[
	\sqrt{d A \epsilon k} > \sqrt{\frac{d A \epsilon k}{3}}
\]
\[
	1 > \frac{\sqrt{3}}{3}
\]

Tato nerovnice platí, tedy nám zbývá jen dosadit tuto hodnotu náboje do vzorce pro napětí:

\[
	U_{max} = \frac{\sqrt{\frac{d A \epsilon k}{3}} \cdot d A \epsilon k - \left(\sqrt{\frac{d A \epsilon k}{3}}\right)^3}{A^2 \epsilon^2 k}
	= \left(\sqrt{\frac{d A \epsilon k}{3}} \cdot d A \epsilon k\right) \cdot \frac{1 - \frac{1}{3}}{A^2 \epsilon^2 k}
	= \frac{2 d \cdot \sqrt{d A \epsilon k}}{3 \sqrt{3} \cdot A \epsilon}
\]

Tím jsme nalezli maximální možné napětí, který může mít tento rezistor.

\end{document}
