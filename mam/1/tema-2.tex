\documentclass{fkssolpub}

\usepackage[czech]{babel}
\usepackage{fontspec}
\usepackage{fkssugar}
\usepackage{amsmath}

\author{Ondřej Sedláček}
\school{Gymnázium Oty Pavla} 
\series{1}
\problem{2} 

\begin{document}

\section{Problém 1}

Jako první zmíním typické příklady zobrazení, které nemůžou být lineární. Nejjednodušší příklad nelineárního zobrazení je posunutí. U ní lze její nelinearita dokázat jednodušše. Otočení ocasu je taky nelineární, protože by muselo nutně dojít také k posunutí. Nelinearitu posunutí si dokážeme.

Nechť je posunutí zobrazení $f(\mathbf{x}) = \mathbf{x} + \mathbf{c}$, kde $\mathbf{c} \neq \mathbf{0}$ je konstanta. Pro lineární zobrazení platí $f(\mathbf{x} + \mathbf{y}) = f(\mathbf{x}) + f(\mathbf{y})$, proto za předpokladu, že posunutí je lineární zobrazení, musí platit $\mathbf{x} + \mathbf{y} + \mathbf{c} = \mathbf{x} + \mathbf{y} + 2 \mathbf{c}$. To však nemůže platit, čímž jsme dokázali, že se nejedná o lineární zobrazení.

Dále ukážu zobrazení, které si můžeme představit jako úpravu, když přehneme průhlednou fólii s obrázkem podél $x$-ové osy, tedy zobrazení, které všechny vektory se zápornou $y$-souřadnicí překlopí přes $x$-ovou osu. Toto zobrazení můžeme vyjádřit jako $f(\mathbf{x}) = (\mathbf{x}_1, |\mathbf{x}_2|)$. A protože absolutní hodnota není lineární funkce, nemůže tato funkce být lineární zobrazení. Dalšími příklady můžou být nějaké kvadratické a jiné nelineární funkce.

Zbytek zobrazení zmíněných zadání splňují podmínky pro lineární zobrazení, a to $f(\mathbf{x} + \mathbf{y}) = f(\mathbf{x}) + f(\mathbf{y})$ a $f(a \mathbf{x}) = a f(\mathbf{x})$. Tyto vlastnosti říkají, že počátek zůstane vždy na stejném místě a že při přičtení stejného vektoru se vektor změní stejně.

\section{Úloha 2}

Výsledky jsou vypsané postupně z leva doprava po řádcích.

\begin{equation}
	\begin{pmatrix}
		24 & -60 \\
		-4 & -90 \\
	\end{pmatrix}
\end{equation}
\begin{equation}
	\begin{pmatrix}
		1 & 0 & 0 & 0 & 0 \\
		2 & 1 & 0 & 0 & 0 \\
		3 & 2 & 1 & 0 & 0 \\
		4 & 3 & 2 & 1 & 0 \\
		5 & 4 & 3 & 2 & 1 \\
	\end{pmatrix}
\end{equation}
\begin{equation}
	\begin{pmatrix}
		18 & -13 \\
		39 & 16  \\
	\end{pmatrix}
\end{equation}
\begin{equation}
	\begin{pmatrix}
		3  & 1  & 9  & 0 & 4  \\
		8  & 6  & 19 & 0 & 14 \\
		13 & 11 & 29 & 0 & 24 \\
		18 & 16 & 39 & 0 & 34 \\
		23 & 21 & 49 & 0 & 44 \\
	\end{pmatrix}
\end{equation}
\begin{equation}
	\begin{pmatrix}
		21 & 12 \\
		31 & 12 \\
		40 & 57 \\
	\end{pmatrix}
\end{equation}
\begin{equation}
	\begin{pmatrix}
		6  & 7  & 8  & 9  & 10 \\
		42 & 44 & 46 & 48 & 50 \\
		1  & 2  & 3  & 4  & 5  \\
		11 & 12 & 13 & 14 & 15 \\
		11 & 12 & 13 & 14 & 15 \\
	\end{pmatrix}
\end{equation}

\section{Úloha 3}

Pro vyřešení této rovnice nám stačí obě rovnice vynásobit takovou inverzní maticí, abychom pak dostali explicitní vzorec pro matici $\mathbf{A}$. To si můžeme dovolit, jelikož tyto matice nejsou singulární.

\[
	\begin{pmatrix}
		1 & 2 \\
		3 & 4 \\
	\end{pmatrix} \cdot \mathbf{A} = \begin{pmatrix}
		5  & 13 \\
		19 & 25 \\
	\end{pmatrix}
\]

\[
	\begin{pmatrix}
		1 & 2 \\
		3 & 4 \\
	\end{pmatrix}^{-1} \cdot
	\begin{pmatrix}
		1 & 2 \\
		3 & 4 \\
	\end{pmatrix} \cdot \mathbf{A} = \begin{pmatrix}
		1 & 2 \\
		3 & 4 \\
	\end{pmatrix}^{-1} \cdot
	\begin{pmatrix}
		5  & 13 \\
		19 & 25 \\
	\end{pmatrix}
\]
\[
	\mathbf{A} = \begin{pmatrix}
		1 & 2 \\
		3 & 4 \\
	\end{pmatrix}^{-1} \cdot
	\begin{pmatrix}
		5  & 13 \\
		19 & 25 \\
	\end{pmatrix} =
	\begin{pmatrix}
		9  & -1 \\
		-2 & 7  \\
	\end{pmatrix}
\]

A to je tedy řešení.

\section{Úloha 4}

Řešením je $\mathbf{A} \in \mathbb{R}^{m \times k}$, $\mathbf{B} \in \mathbb{R}^{k \times l}$, $\mathbf{C} \in \mathbb{R}^{m \times l}$, $\mathbf{D} \in \mathbb{R}^{n \times l}$, $\mathbf{E} \in \mathbb{R}^{n \times m}$, $\mathbf{F} \in \mathbb{R}^{n \times l}$.

\section{Úloha 5}

Když si tuto úlohu vyjádříme jako rovnici:

\[
	\mathbf{X} \cdot
	\begin{pmatrix}
		-7 & 9  \\
		4  & -5 \\
	\end{pmatrix}
	=
	\begin{pmatrix}
		5  & -4 \\
		-3 & 2  \\
	\end{pmatrix}
\]

Tak tuto rovnici můžeme upravit na:

\[
	\mathbf{X}
	=
	\begin{pmatrix}
		5  & -4 \\
		-3 & 2  \\
	\end{pmatrix} \cdot
	\begin{pmatrix}
		-7 & 9  \\
		4  & -5 \\
	\end{pmatrix}^{-1}
	=
	\begin{pmatrix}
		9  & 17  \\
		-7 & -13 \\
	\end{pmatrix}
\]

\end{document}
