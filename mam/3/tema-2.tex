\documentclass{fkssolpub}

\usepackage[czech]{babel}
\usepackage{fontspec}
\usepackage{fkssugar}
\usepackage{amsmath}
\usepackage{graphicx}

\newcommand{\dd}{\mathrm{d}}
\renewcommand{\angle}{\sphericalangle}
\newcommand*{\vertbar}{\rule[-1ex]{0.5pt}{2.5ex}}
\newcommand{\R}{\mathbb{R}}
\newcommand{\Q}{\mathbb{Q}}

\author{Ondřej Sedláček}
\school{Gymnázium Oty Pavla} 
\series{3}
\problem{2} 

\begin{document}

\section{Úloha 1}

Abychom našli požadované rovnice, musíme najít řešení soustavy rovnic níže:

\[
  6 a - 3 b + 2 c = 0
\]
\[
  -4 a + 5 b - 5 c = 0
\]
\[
  0 a - 9 b + 11 c = 0
\]

Pro to nám stačí najít RREF matice níže:

\[
  \left(
  \begin{array}{ccc|c}
    6 & -3 & 2 & 0 \\
    -4 & 5 & -5 & 0 \\
    0 & -9 & 11 & 0 \\
  \end{array}
  \right) \sim \left(
  \begin{array}{ccc|c}
    1 & 0 & -\frac{5}{18} & 0 \\
    0 & 1 & -\frac{11}{9} & 0 \\
    0 & 0 & 0 & 0 \\
  \end{array}
  \right)
\]

Tudíž $c \neq 0$ je parametr a $b = \frac{11}{9} c$ a $a = \frac{5}{18} c$. Z toho získáme:

\[
  \frac{5}{18} c x + \frac{11}{9} c y + c z = 0
\]
\[
  \frac{5}{18} x + \frac{11}{9} y + z = 0
\]

Získali jsme tedy rovnici roviny, ve které vektory leží.

\section{Úloha 2}

Víme, že když jsou generátory lineárně závislé, pak jeden z vektorů $\mathbf{v_i}$ lze vyjádřit jako:

\[
  \mathbf{v_i} = \alpha_1 \mathbf{v_1} + \dots + \alpha_{i - 1} \mathbf{v_{i - 1}} + \alpha_{i + 1} \mathbf{v_{i + 1}} + \dots + \alpha_k \mathbf{v_k}
\]

Kde alespoň některé koeficienty $\alpha_1, ..., \alpha_k$ jsou nenulové. Pak můžeme vyjádřit nulový vektor i jako:

\[
  \mathbf{0} = \alpha_1 \mathbf{v_1} + \dots + \alpha_{i - 1} \mathbf{v_{i - 1}} - \mathbf{v_i} + \alpha_{i + 1} \mathbf{v_{i + 1}} + \dots + \alpha_k \mathbf{v_k}
\]

Tedy lineární kombinaci výše můžeme přičítat k vektoru ve $\text{span} G$ a dostaneme tentýž vektor. Tím jsme dokázali opačnou implikaci.

\section{Úloha 3}

Pro první množinu můžeme využít toho, že vektorový součin je definovaný ve vektorovém prostoru $\mathbb{R}^3$. Tedy řešením je:

\[
  M = \left\{
    \begin{pmatrix}
      2 \\ -3 \\ 6
    \end{pmatrix}
    ,
    \begin{pmatrix}
      -1 \\ 2 \\ -4
    \end{pmatrix}
    ,
    \begin{pmatrix}
      2 \\ -3 \\ 6
    \end{pmatrix} 
    \times
    \begin{pmatrix}
      -1 \\ 2 \\ -4
    \end{pmatrix}
    \right\} =
  \left\{
    \begin{pmatrix}
      2 \\ -3 \\ 6
    \end{pmatrix}
    ,
    \begin{pmatrix}
      -1 \\ 2 \\ -4
    \end{pmatrix}
    ,
    \begin{pmatrix}
      0 \\ 2 \\ 1
    \end{pmatrix} 
    \right\}
\]

U druhé množiny dokážeme, že můžeme přidat vektory $(0,0,1,0)$ a $(0,0,0,1)$. Tyto dva vektory jsou zřejmě na sobě nezávislý. Pak pokud je první vektor lineárně závislý na vektorech množiny $M$, musí matice níže mít řešení:

\[
  \left(
  \begin{array}{cc|c}
    2 & 2 & 0 \\
    4 & 2 & 0 \\
    8 & 3 & 1 \\
    1 & -10 & 0 \\
  \end{array}
  \right) \sim \left(
  \begin{array}{cc|c}
    2 & 2 & 0 \\
    2 & 0 & 0 \\
    8 & 3 & 1 \\
    1 & -10 & 0 \\
  \end{array}
  \right) \sim \left(
  \begin{array}{cc|c}
    1 & 0 & 0 \\
    0 & 2 & 0 \\
    0 & 3 & 1 \\
    0 & -10 & 0 \\
  \end{array}
  \right) \sim \left(
  \begin{array}{cc|c}
    1 & 0 & 0 \\
    0 & 1 & 0 \\
    0 & 0 & 1 \\
    0 & 0 & 0 \\
  \end{array}
  \right)
\]

Tedy vektor $(0,0,1,0)$ je opravdu nezávislý. Stejně to uděláme pro druhý vektor:

\[
  \left(
  \begin{array}{cc|c}
    2 & 2 & 0 \\
    4 & 2 & 0 \\
    8 & 3 & 0 \\
    1 & -10 & 1 \\
  \end{array}
  \right) \sim \left(
  \begin{array}{cc|c}
    1 & 0 & 0 \\
    0 & 2 & 0 \\
    0 & 3 & 0 \\
    0 & -10 & 1 \\
  \end{array}
  \right) \sim \left(
  \begin{array}{cc|c}
    1 & 0 & 0 \\
    0 & 1 & 0 \\
    0 & 0 & 0 \\
    0 & 0 & 1 \\
  \end{array}
  \right) 
\]

Tedy jsme ukázali, že oba tyto vektory jsou nezávislé na zbytku, tedy je lze doplnit do množiny $M$.

\section{Úloha 4}

Víme, že vektor $\mathbf{w}$ je lineární kombinací generátorů, tedy:

\[
  \mathbf{w} = \alpha_1 \mathbf{v_1} + \alpha_2 \mathbf{v_2} + \dots +\alpha_n \mathbf{v_n}
\]

kde existuje $\alpha_k \neq 0$. Pak když vyměníme $\mathbf{v_k}$ za $\mathbf{w}$, umíme vyjádřit $\mathbf{v_k}$ jako:

\[
  \mathbf{v_k} = -\frac{\alpha_1}{\alpha_k} \mathbf{v_1} - \frac{\alpha_2}{\alpha_k} \mathbf{v_2} - \dots - \frac{\alpha_n}{\alpha_k} \mathbf{v_n} - \frac{1}{\alpha_k} \mathbf{w}
\]

Tedy když máme nějakou lineární kombinaci vektoru v $P$, stačí nám do ní dosadit $\mathbf{v_k}$, abychom dostali lineární kombinaci s novými generátory, tedy nová množina generátorů generuje $P$.

\section{Úloha 5}

Pokud budeme vektory $\mathbf{u_1}$, $\mathbf{u_2}$, ..., $\mathbf{u_m}$ přidávat po jednom, můžeme použít lemma o výměně. Z lemmatu o výměně víme, že v každém kroku, kdy provádíme výměnu s nějakým generátorem, existuje generátor, se kterým lze vektor vyměnit, a tedy tuto výměnu provedeme. Zbývá tedy dokázat, že vždy budeme moci tyto vektory vyměnit s původními generátory.

Pokud by v nějakém kroku nastalo, že by nějaký vektor $\mathbf{u_i}$ šel vyměnit jen s vektory $\mathbf{u_1}$, ..., $\mathbf{u_{i - 1}}$, znamenalo by to, že $\mathbf{u_i}$ lze vyjádřit jako lineární kombinace těchto vektorů. To je však spor s tím, že tyto vektory jsou lineárně nezávislé, a tedy vždy budeme moci tyto vektory vyměnit s původními generátory. Tím jsme dokázali Steinitzovu větu o výměně.

\section{Úloha 6}

Většinou budeme znát vyjádření bazických vektorů v kanonické bázi, proto tedy budeme schopni pro báze $B$,$C$ snadno zkonstruovat matice $_{\text{kan}}[\mathbf{id}]_B$ a $_{\text{kan}}[\mathbf{id}]_C$, jelikož sloupce těchto matic odpovídají bazickým vektorům báze $B$, resp. $C$. A protože $_{\text{kan}}[\mathbf{id}]_B \cdot _{B}[\mathbf{id}]_{\text{kan}} = _{\text{kan}}[\mathbf{id}]_B \cdot (_{\text{kan}}[\mathbf{id}]_B)^{-1} = \mathbf{I}$, umíme získat převody z kanonické báze na báze $B$ nebo $C$ inverzí převodů z bází $B$,$C$ do kanonické. Tedy:

\[
  _B[\mathbf{id}]_C = _B[\mathbf{id}]_{\text{kan}} \cdot _{\text{kan}}[\mathbf{id}]_C = 
\left(
  \begin{array}{cccc}
    \vertbar & \vertbar &        & \vertbar \\
    \mathbf{b_{1}}    & \mathbf{b_{2}}    & \ldots & \mathbf{b_{n}}    \\
    \vertbar & \vertbar &        & \vertbar 
  \end{array}
\right)^{-1} \cdot 
\left(
  \begin{array}{cccc}
    \vertbar & \vertbar &        & \vertbar \\
    \mathbf{c_{1}}    & \mathbf{c_{2}}    & \ldots & \mathbf{c_{n}}    \\
    \vertbar & \vertbar &        & \vertbar 
  \end{array}
\right)
\]
\[
  _C[\mathbf{id}]_B = _C[\mathbf{id}]_{\text{kan}} \cdot _{\text{kan}}[\mathbf{id}]_B = 
\left(
  \begin{array}{cccc}
    \vertbar & \vertbar &        & \vertbar \\
    \mathbf{c_{1}}    & \mathbf{c_{2}}    & \ldots & \mathbf{c_{n}}    \\
    \vertbar & \vertbar &        & \vertbar 
  \end{array}
\right)^{-1}
\cdot 
\left(
  \begin{array}{cccc}
    \vertbar & \vertbar &        & \vertbar \\
    \mathbf{b_{1}}    & \mathbf{b_{2}}    & \ldots & \mathbf{b_{n}}    \\
    \vertbar & \vertbar &        & \vertbar 
  \end{array}
\right)
\]

\section{Úloha 7}

Nechť $\mathbf{u} = (2, 0, -1)^{\top}$. Jedna z podprostorů je určitě $\text{span}(\{\mathbf{u}\})$, což je taky jediný jednodimenzionální podprostor obsahující tento vektor. Pak pro dvojrozměrné pak můžeme každý podprostor vyjádřit jako $\text{span}(\{\mathbf{u}, \mathbf{v}\})$, kde $\mathbf{v} \in \mathbb{R}^3 \setminus \{k \mathbf{u}; k \in \mathbb{R}\}$, tedy vektor $\mathbf{v}$ je jakýkoli vektor, který je lineárně nezávislý na $\mathbf{u}$. Zbývají teď třírozměrné podprostory, z nichž jediná je $\mathbb{R}^3$.

\section{Úloha 8}

Nejprve dokážu, že celá čísla $\mathbb{Z}$ se sčítáním a násobením nejsou těleso. Pokud by se totiž jednalo o těleson, pak pro každé celé číslo $x$ existuje takové $y$, že $x \cdot y = 1$, což ale zřejmě není pravda (stačí zkusit třeba za $x$ dosadit 2). Proto $\mathbb{Z}^n$ nad $\mathbb{Z}$ není lineární prostor.

Protože množiny $\Q^n$ nad $\Q$ a $\R^n$ nad $\Q$ splňují všechna pravidla lineárního prostoru a její operace jsou uzavřené, jedná se o lineární prostory. Avšak množina $\Q^n$ nad $\R$ není uzavřená množina, proto se o lineární prostor nejedná (stačí např. vynásobit vektor číslem $\pi$).



\end{document}
